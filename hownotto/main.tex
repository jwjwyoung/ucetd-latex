 \documentclass[sigconf]{acmart}

\usepackage{booktabs} % For formal tables
\usepackage{verbatim}
\usepackage{soul}
\usepackage{color,calc}
\PassOptionsToPackage{bookmarks=false}{hyperref}
\usepackage{listings}
\usepackage{tcolorbox}
\usepackage{comment}
\usepackage{xspace}
\usepackage{subcaption}
\usepackage[inline]{aplcomments}
\usepackage{multirow}
\usepackage{tikz}
\usepackage{color, colortbl}
\usepackage{flushend}
\usepackage[font=small,skip=0pt]{caption}

%\usepackage[disabled]{aplcomments}
\usepackage[labelformat=simple]{subcaption}
\renewcommand\thesubfigure{(\alph{subfigure})}
% Copyright
%\setcopyright{acmcopyright}
%\setcopyright{acmlicensed}
\setcopyright{acmlicensed}
%\setcopyright{usgov}
%\setcopyright{usgovmixed}
%\setcopyright{cagov}
%\setcopyright{cagovmixed}

% DOI
\acmDOI{10.1145/3180155.3180194}

% ISBN
\acmISBN{978-1-4503-5638-1/18/05}
%Conference
\acmConference[ICSE'18]{ICSE '18: 40th International Conference on
Software Engineering}{May 27-June 3, 2018}{Gothenburg, Sweden} 
\acmYear{2018}
\copyrightyear{2018}

\makeatletter
\def\subsubsection{\@startsection{subsubsection}{3}%
  \z@{.5\linespacing\@plus.7\linespacing}{.1\linespacing}%
  {\normalfont\itshape}}
\makeatother 
\acmPrice{15.00}

\newcommand*\circled[1]{\tikz[baseline=(char.base)]{
            \node[shape=circle,draw,inner sep=1.5pt] (char) {#1};}}
 \newcommand*\smallcircled[1]{\tikz[baseline=(char.base)]{
            \node[shape=circle,draw,inner sep=1pt] (char) {#1};}}
\newcommenter{shan}{1.0,0.0,0.0}
\newcommenter{junwen}{0.0,1.0,0.0}
\newcommenter{cong}{0.0,0.8,0.8}
\newcommenter{alvin}{0.0,1.0,1.0}
\newcommand{\secref}[1]{Section~\ref{s:#1}}
\newcommand{\figref}[1]{Figure~\ref{f:#1}}     % for use in text

\newcommand{\numissues}{140\xspace}
\newcommand{\numacissues}{64\xspace}
\newcommand{\numpactions}{40\xspace}
\newcommand{\numconfirmedapi}{53\xspace}
\newcommand{\numfixedapi}{29\xspace}
\newcommand{\numconfirmedpaissues}{14\xspace}
\newcommand{\numfixedpaissues}{7\xspace}

\newcommand{\maxspeedup}{39\xspace}

\newcommand{\eoebefore}{4.17\xspace}
\newcommand{\eoeafter}{0.69\xspace}
\newcommand{\serverbefore}{3.57\xspace}
\newcommand{\serverafter}{0.49\xspace}

\newcommand{\seclabel}[1]{\label{s:#1}}

\newcommand{\lstlabel}[1]{\label{lst:#1}}
\newcommand{\figlabel}[1]{\label{f:#1}}



\newcommand{\listref}[1]{Listing~\ref{#1}}
\definecolor{LightCyan}{rgb}{0.88,1,1}
%\definecolor{Gray}{gray}{0.9}
\definecolor{Gray}{rgb}{0.6,0.6,0.6}
\lstset{language=Ruby,
                captionpos=b,
                xleftmargin=.01in,
                keywordstyle=\color{blue},
								escapeinside={(*}{*)},
                %keywordstyle=\color{red},
                %morekeywords={select,update,insert},
                %deletekeywords={if,else},
                %stringstyle=\color{red}\ttfamily,
								showstringspaces=false,
								breaklines=true,
								numbers=none,
								numbersep=3pt, 
								tabsize=2,  
								numberstyle=\tiny\color{gray},
                commentstyle=\color{brown},
								basicstyle=\small\ttfamily,
                escapeinside={@}{@}
}

\lstset{language=SQL,
        captionpos=b,
        xleftmargin=.01in,
        keywordstyle=\color{blue},
        showstringspaces=false,
		breaklines=true,
		numbers=none,
		numbersep=3pt, 
		tabsize=2,  
		numberstyle=\tiny\color{gray},
        commentstyle=\color{brown},
		basicstyle=\small\ttfamily,
        escapeinside={@}{@}
}

\begin{document}
\title{How {\em not} to structure your database-backed web applications:\\
a study of performance bugs in the wild 
}
\titlenote{\tt https://hyperloop.cs.uchicago.edu}

\author{Junwen Yang\\Pranav Subramaniam\\Shan Lu}
\affiliation{
  \institution{University of Chicago \\
  { {\tt \{junwen, psubramaniam, shanlu\}@uchicago.edu} }
  }
}
\author{Cong Yan\\Alvin Cheung}
\affiliation{
  \institution{University of Washington\\
  { {\tt\{congy, akcheung\}@cs.washington.edu}}
  }
}




\begin{abstract}
Many web applications use databases for persistent data storage, and using Object Relational Mapping (ORM) frameworks is a common way to develop such database-backed web applications.
Unfortunately, developing efficient ORM applications is challenging, as the ORM framework hides the underlying database query generation and execution. This problem is becoming more severe as these applications need to process an increasingly large amount of persistent data. 
Recent research has targeted specific aspects of performance problems in ORM applications. However, 
there has not been any systematic study to identify common performance anti-patterns in real-world
such applications, how they affect resulting application performance, and remedies for them.
%it is still unclear what are all the performance anti-patterns that are common in real-world ORM applications historically and currently, how they are affecting the performance and scalability of today's ORM applications, and how they might be effectively tackled. 

In this paper, we try to answer these questions through a comprehensive study of 12 representative real-world ORM applications. We generalize 9 ORM performance anti-patterns from more than 200 performance issues that we obtain by studying their bug-tracking systems and
profiling their latest versions. To prove our point, we manually fix 64 performance issues in their latest versions and obtain a median speedup of 2$\times$ (and up to $\maxspeedup\times$ max) with fewer than 5 lines of code change in most cases. 
Many of the issues we found have been confirmed by developers, and we have implemented ways to identify other code fragments with similar issues as well.
\end{abstract}

%
% The code below should be generated by the tool at
% http://dl.acm.org/ccs.cfm
% Please copy and paste the code instead of the example below. 
%
\begin{CCSXML}
<ccs2012>
<concept>
<concept_id>10011007.10010940.10011003.10011002</concept_id>
<concept_desc>Software and its engineering~Software performance</concept_desc>
<concept_significance>500</concept_significance>
</concept>
<concept>
<concept_id>10011007.10011074.10011111.10011696</concept_id>
<concept_desc>Software and its engineering~Maintaining software</concept_desc>
<concept_significance>500</concept_significance>
</concept>
</ccs2012>
\end{CCSXML}

\ccsdesc[500]{Software and its engineering~Software performance}

\keywords{performance anti-patterns, Object-Relational Mapping frameworks, database-backed applications, bug study}

% akcheung: add page numbers and remove copyright and title texts
\settopmatter{printfolios=true}
\settopmatter{printacmref=false} % Removes citation information below abstract
%\renewcommand\footnotetextcopyrightpermission[1]{} % removes footnote with conference information in first column
\pagestyle{plain} % removes running headers

\maketitle
\label{sec:intro}
Modern web applications often use database engines to 
manage a large amount of user data, such as user profiles in social network applications and transaction
records in on-line shopping platforms~\cite{webapp}.  
The schema of such data goes through changes, such as table renaming, column
deletion, and others, for better performance or functionality when an application evolves \cite{wang2017verifying}.
Unfortunately, it is difficult for developers to keep their code 
consistent with database schema changes all the time, a task we
refer to as \textit{schema-related code refactoring}, with any inconsistency leading
to application crashes.

Schema-related refactoring and
traditional refactoring like class renaming share similarities, given that
popular Object Relational Mapping (ORM) frameworks, such as Rails \cite{rails},  
Django \cite{django}, and Hibernate \cite{hibernate},
allow database data to be
updated and retrieved in an object-oriented way---the name of a database table corresponds to the name of a model class
and the names of table columns are the same as class fields.

%\shan{TODO: we may need to expand the onebody example}
However, they also differ in various aspects, due to the unique nature of persistent data,
as we elaborate below\footnote{Our discussion generally applies to
all web applications developed upon ORM frameworks, although our
examples use Ruby on Rails applications.}.



%Comparing with traditional software, database-backed web applications face the unique code-maintenance challenge of making code changes consistent with database schema changes. Because web applications are often constructed using Object Relational Mapping (ORM) frameworks allowing the properties and relationships of the objects in an application to be easily stored and retrieved from a database in an object-oriented way, which can go beyond the form of class name and field name as used in traditional software, e.g.,  in line 2 in listing~\ref{onebody-query}, the \texttt{sequence} column of \texttt{people} table is used as a parameter of \texttt{order} function call in the form of string. 



% The class definition in database-backed web applications  will not explicitly declare its fields since database columns  will be mapped to the fields by ORM frameworks implicitly. 
% ORM frameworks will implicitly map database columns to the object's class. 
 


% \begin{lstlisting}[float=t,language=Ruby,label={onebody-query}, caption=Inconsistent code from Onebody]
% app/controllers/checkin/families_controller.rb
%     # select * from people where family_id = ? 
%     # and is_deleted = ? order by sequence
% 1   @family.people.undeleted.order('sequence')

% app/models/person.rb
% 2   person.sequence = 1
% \end{lstlisting}
% For example, in Spree \cite{spree}, an online-shopping application, a table named orders is used to 
% keep the order information. In one version, 
% developers renamed a column in this table from \texttt{guest\_token} to \texttt{token} since the column starts to store \texttt{tokens} used in login cookies for not only guest but also general users\shan{what does token mean? what does general token mean?}, it has to change all the references to the old name \texttt{guest\_token} to the new name \texttt{token}, which results in 284 lines of code change across 32 files. 
%It's challenging for developers to manually refactor the application code based on the schema change. 

\textit{How is schema defined?} Different from a regular class whose 
field names and field types are defined by its class
declaration, a model class's structure has to match its corresponding database table that is created once at an application's installation or upgrade.
%with its data retrieved into objects at run time.
In fact, in some ORM frameworks like Rails, persistent fields of a model class are \textit{not} declared in its class
definition and are instead automatically mapped by Rails from the corresponding table schema, which is 
%Consequently, although every table (e.g., \texttt{people}) corresponds to a class
%with a singularized name (e.g., \texttt{Person}), the schema, including table names, column
%names and types, and other information, has to be 
defined through ORM
APIs like \texttt{create\_table} in Rails or
\texttt{CreateModel} in Django in a type of files called
\textit{migration files},
as shown in Listing \ref{migration}. 
%In fact, the corresponding model class definition needs not contain definitions of column fields.

\textit{How is schema changed?} 
%One cannot change the
%schema (e.g., a column name) by changing the class definition. 
Schema changes are expressed
through ORM APIs in migration files (e.g., line 4 in Listing 
\ref{migration}
renames the column \texttt{sequence} in table \texttt{people}), 
which informs the web application about how to 
update its database during installation and upgrade.
In an ORM framework like Rails, schema changes cannot be seen in
class definitions.

\lstinputlisting[float=t, basicstyle=\footnotesize\ttfamily, label={migration}, caption={Migration files from Onebody},language=Ruby]{migration.rb}

\lstinputlisting[float=t,basicstyle=\footnotesize\ttfamily, label={onebody-query}, caption={Inconsistent code from Onebody},language=Ruby]{onebody-query.rb}

\textit{What code refactoring is needed?} Following a schema
change in the migration file, corresponding references in the
application need to change. Some of these are just
class or field renaming like in line 2 of Listing \ref{onebody-query}, while some require changing ORM APIs' parameters like in line 1 of Listing \ref{onebody-query}.
%and hence requires ORM knowledge.

For example, developers of Onebody~\cite{onebody}, a popular social network application, used 
a table \texttt{people} to keep user information. In one commit, they renamed the \texttt{sequence} column in \texttt{people} to \texttt{position} (line 4 in Listing \ref{migration}). In the same commit, they
correctly updated the reference to \texttt{sequence} in 6 places across 4 files like the one in line 2 of Listing \ref{onebody-query}, but forgot to change the other 5 places, 
such as the parameter reference shown in line 1 of Listing
\ref{onebody-query}. This inconsistency caused web users to
suffer from webpage crashes~\cite{onebodyissue672}.
%\url{https://github.com/seven1m/onebody/issues/672}}
%\shan{the issue report does not read like application crash.}

%https://github.com/spree/spree/pull/8826/files

% Moreover, the challenge is escalated by the prevalent usage of Object-Relational Mapping (ORM) frameworks, such as Ruby on Rails~\cite{rails}, Django for Python~\cite{django}, and Hibernate for Java~\cite{django}. Because they provide a convenient way for developers to evolve their database schema over time using object-oriented code without issuing `ALTER TABLE' SQL commands to database, which eases the efforts and results in more frequent schema change. 
% \shan{Do you have evidence that there are more frequent schema changes in ORM applications?
% If not, I don't think we can claim it. I can imagine maybe it is more error prone, because
% schema changes and declaration are in different files from files that use the tables and columns,
% so that it is more isolated ... I don't know ...} 
% \junwen{I have found that in an empirical study on web application not using ORM framework~\cite{qiu2013empirical}, the average schema change per release is 60, so we cannot draw the conclusion that orm schema changes faster. But their way of calculating change is to check every revision instead of every release. We may change our counting methodology and regenerate the result. 
% }


%TODO:\shan{There needs to be a clearer definition of what are code-schema inconsistencies.}
%TODO:\shan{can some of the inconsistency get exposed by unit testing? }
% Inconsistency between application code and its data schema could cause web-page crashes 
% with database exceptions thrown 
% Different from traditional software, the object class's fields in database-backed web applications are not declared explicitly in the class itself but mapped to the database columns by ORM frameworks implicitly
 
% Thus, existing refactoring tools that tackle method or field renaming are not able to refactor the database-backed web applications.

Recent work motivated tool support for schema-related refactoring  \cite{wang2017verifying} and proposed techniques to synthesize updates to a list of 
SQL queries given the old schema and the new schema written in SQL
\cite{wang2019synthesizing}. Although inspiring, it does not directly help
many web applications, whose schema changes
and database operations are expressed in ORM APIs, rarely if ever in raw SQL.

%As a result, cross-stack analysis which combines the database information with application code is necessary to detect and fix the inconsistency  between application code and schema change. 
%Existing work MIGRATOR~\cite{wang2019synthesizing} can automatically synthesize database program after schema refactor, however it can only work on database programs written in SQL. 
% Also, MIGRATOR is guessing value correspondence from the name of table and column of two schema\shan{I don't understand}, which is possible to cause inaccurate mapping and generate incorrect refactoring. Also, MIGRATOR is not able to handle deletion change. \shan{I don't understand.} 



This paper presents {}, a tool that uses ORM-aware static analysis to help schema-related code refactoring in web applications
written in Rails \cite{rails} and Django \cite{django}, two popular web frameworks.
%\shan{can we say 'most' popular? can we say web application frameworks, instead of ORM frameworks to make it sound more general?}. \junwen{According to the data \url{https://www.statista.com/statistics/1124699/worldwide-developer-survey-most-used-frameworks-web/}, Django and Rails are not the most popular web app frameworks. But based on data we collected on github before, for apps with more than 100 starts, the most popular is Rails and then django}
Given two versions of a web application, 
\ETool{} analyzes and identifies schema changes from
migration files, searches for
any code inconsistent with the new schema,
and generates warnings and patches accordingly. %\shan{Why do we need the old code version? If I just give you one version,can you identify code--schema inconsistency?} \junwen{Yes, we can. Having the old version schema is to just make the detection even faster by only checking on the changed schema}.

%As an IDE plugin, \ETool{} supports two usage scenarios: 1) it can semi-automate \shan{why semi? what is not automated?} the refactoring for an old application after users make data-schema changes \shan{what type of data schema changes do you support? What is the scope/limitations of your tool?}. 2) it can identify remaining inconsistency between the data schema and the application code, after
%developers commit changes to both data schema and the application code.  %\shan{I can imagine two potential uses of this type of tools: (1) once a data schema change is made, it will help automate or semi-automate all the needed code changes; (2) given a data schema change AND changed software, it checks if there are inconsistency between the data schema and the software. Does your tool only offer functionality-2? Or do you offer both? I guess you offer both, if so, you should rephrase the usage flow and functionality description of your tool.} 
%This work makes the following contributions:


%\shan{What type of guarantees you can offer here? Do you guarantee that you will generate perfectly correct and complete refactoring?} 

To ease its adoption, we have integrated \ETool into the popular 
Visual Studio Code IDE~\cite{vscodepop} as a plugin. Web developers can use this plugin to
guide their schema-related refactoring or to look for
schema--code inconsistency bugs.

In our evaluation with 12 popular Rails and Django applications, 
\ETool detected \numRailsError schema--code inconsistencies 
%in 72 files 
caused by 35 schema changes in the past.
%and 25 warnings caused by 4 schema changes in application code that are caused by schema changes.  Among the 80 errors, 
We have reported 11 of them that exist in the latest versions to developers,
and got 10 of them already confirmed and 6 of them already patched based on
our suggestion.
Our examination of the rest \numFixed inconsistencies shows that 
they took many days for developers
to discover and fix.
%they took 378 days and 3 days on average for Rails and Django developers to discover and fix, respectively. %Moreover, \numfixedafterrelease of them are fixed after the version release. 

\ETool's source code is on Github~\cite{sourcecode} and the plugin can be downloaded from Visual Studio Marketplace~\cite{vscodemarketplace}.

% * <congy@uw.edu> 2018-02-06T00:43:47.760Z:
%
% ^.

\label{sec:bg}
%Rails applications are structured based on the model-view-controller
%(MVC) architecture. We illustrate this architecture in Figure~\ref{fig:ormf}. When a client requests for a URL such as {\tt http://foo.com/projects/index/1} \alvin{to make it more real how about change ... to something like foo.com}, Rails maps these requests to the controller action as shown in the Routing rules as shown in the figure \alvin{what does these rules do? is it important to know about them in this paper?}.  The mapped \alvin{?} Controller takes in the parameters from the requests and asks model to retrieve data from database. The returned data then is used to generate view which will be sent to the client side. An action is a member method of a Ruby controller class. There could exist multiple actions in a controller file to handle  different user requests. \alvin{the description here needs to map to the figure. I suggest taking a look at the description in our CIKM paper or the ICSE paper}\cong{I'll revise it.}
\iffalse 
\begin{figure}[ht]
    \centering
    \includegraphics[width=1\columnwidth]{figs/ormf.pdf}
    \vspace{0.05in}
    \caption{Rails application architecture }
    \label{fig:ormf}
    \vspace{-0.2in}
\end{figure}
\fi 

% \subsection{Architecture of web applications} 
% % Rails applications are structured based on the model-view-controller (MVC) architecture. When a client requests for a URL such as {\tt http://foo.com/projects/index/1}, a {\it controller} action ``{\tt projects /index}'' is triggered. This action takes in the parameters from the request (e.g., ``{\tt 1}'' in the URL as {\tt params[:id]}) and interacts with the DBMS by calling the ActiveRecord API implemented by the Rails Object-Relational Mapping (ORM) framework. Rails translates the function calls into SQL queries, whose results are serialized into {\it model} objects (e.g., the {\tt Project} model) and returned to the controller. Then, the returned objects are passed to the {\it view} files to generate a webpage to send back to users. Each model is derived from {\tt ActiveRecord}, and is mapped to a database table by Rails. 
% Applications built using the Ruby on Rails framework are structured using the model-view-controller (MVC) architecture. 
% For example, when a web user submit a form through a URL like {\tt http://foo.com/wikis/new/title=release}, a {\it controller} action ``{\tt wikis/create}'' is triggered. This action takes in the parameters from the request (e.g., ``{\tt release}'' in the URL as {\tt params[:title]}) and interacts with the database by calling the ActiveRecord API implemented by the Rails Object-Relational Mapping (ORM) framework. Rails translates ActiveRecord function calls into SQL queries (a write query in this case), whose results are then serialized into {\it model} objects (e.g., the {\tt Wiki} model) and returned to the controller. The returned objects are then passed to the {\it view} files to generate a webpage that is sent back to users. Each model is derived from {\tt ActiveRecord}, and is mapped to a database table by Rails. 
% %The data retrieval is done by translating a {\tt ActiveRecord} API call into a database query. The returned database tuples are converted into objects and sent to a {\it view} be rendered. 
% A view file (ends with {\tt .erb} or {\tt .haml}) usually involves multiple languages including HTML, JavaScript, and Ruby. 
% %The ruby code can dynamically decide which element to show. 

\subsection{Constraints in web applications}
\label{sec:back_constraints}



%\shan{this sub-section might better belongs to the background or we can have a taxonomy section or something like that} 

We roughly categorize data constraints into three types based on where they are checked and specified
as shown in Table~\ref{table:constraintdeftax}.

%\shan{XXX what exactly is the definition of ``different layers''? Is it based on where the checking is conducted?}

\textbf{Front-end constraints.} 
Developers can use regular expressions to specify constraints
about a particular HTML data-field inside a view file, such as the {\tt pattern=`.+'} for the {\tt title}
field %in {\tt views/wiki/new.html.erb} 
in Figure~\ref{fig:crossstack}. The majority of such constraints are related to persistent data maintained by the database.
 
Such constraints are checked when the user submits a web form. Failure to validate will cause the form submission to fail, with an error message specified by developers shown next to the corresponding HTML field, with all the previously filled contents remain on the page.
%with other filled contents on the page. \alvin{seems a bit too detail? maybe just say `Failure to validate will cause the form submission to fail with an error shown on the page'?}

%\shan{other the already filled content on the page will still be there, right? XXX}.
  %\shan{according to your figure, the error message should be ``invalid title'', right? XXX}. 
%\utsav{We are only considering standard HTML constraints (e.g. min/maxlength, pattern, etc.), and we might want to investigate how frequently constraints are instead expressed as ruby code in the view layer, which I think may be possible in templates. } \junwen{TODO: collect the stats about the if checking in the view file in the non-API constraints}

\textbf{Application constraints.} 
Rails developers use {\it validation functions} to specify constraints of data fields in model classes.
Similar mechanisms exist in other ORM frameworks, such as validator functions in Django \cite{django-annotation}, and validator annotations in Hibernate \cite{hibernate-annotation}.

\begin{table}
\centering
\setlength{\tabcolsep}{1.2pt}  
\caption{Different types of constraints in web apps}
\label{table:constraintdeftax}
\resizebox{.7\columnwidth}{!}{
\begin{tabular}{@{}llll@{}}
\toprule
{\bf Run-time check}               & {\bf Source-code} & {\bf Specification}  & {\bf Specification} \\ 
{\bf location}                     & {\bf location}    & {\bf language}       & {\bf API} \\
\midrule
Front end                          & View  & HTML & Reg. expression           \\ \midrule
\multirow{3}{*}{Application server}& Model & Ruby & Built-in validation API    \\ \cmidrule(l){2-4} 
                                   & Model & Ruby & Custom validation API  \\ \cmidrule(l){2-4} 
                                   & Model/Controller&Ruby & Custom sanity check \\ \midrule
\multirow{2}{*}{Database server}   & Migration files &Ruby& ActiveRecord::Migration API\\ \cmidrule(l){2-4} 
                                   & Migration files &SQL & SQL ALTER TABLE queries   \\ \bottomrule
\end{tabular}
}
 \vspace{-0.2in}
   
\end{table}

% hibernate: https://hibernate.org/validator/documentation/getting-started/
% django: https://docs.djangoproject.com/en/2.2/ref/validators/

A validation function is automatically triggered every time when the application saves an object of the corresponding model class (i.e., when the ORM framework saves the corresponding record into the database).
Validation failure will cause the corresponding form to fail. The error message associated with the validation function will be shown to web users if developers put error checking and error-message display code in the view file.
%, as shown in Figure \ref{fig:error-msg}. 
%TODO. will add back the figure if we have space

\iffalse 
\begin{figure} 
    \centering
    \includegraphics[width=0.9\columnwidth]{figs/error-messages-model.pdf}
    \caption{Error-message rendering in Ror-ecommerce \cite{ror}}
    \label{fig:error-msg}
    \vspace{-0.1in}
\end{figure}
\fi 

\iffalse 
\begin{figure} 
    \centering
    \includegraphics[width=0.7\columnwidth]{figs/custom-validator-example.pdf}
    \caption{Custom-validate code example}
    \label{fig:custom-validator-code}
\end{figure}
\fi  
 
Rails validation
functions include built-in ones, which cover many common constraints like text-field lengths (i.e., {\tt validates\_ length\_of}, as shown in Figure \ref{fig:crossstack}),
content uniqueness ({\tt validates\_ uniqueness\_of}), 
content presence ({\tt validates\_presence\_of}),
%numericality requirement ({\tt validates\_numericality\_of}),
as well as custom ones, where developers express more complicated constraints like keeping a strict order among multiple fields.
%as shown in Figure \ref{fig:custom-validator-code}.
%\utsav{Figure 3 needs to be fixed/reformatted..}

Developers can also constrain a data field through custom sanity checks, although they are uncommon in Rails. 
%These   checks typically are put right after a new set of inputs are obtained or right before saving content to database. 
%How to handle a sanity-check failure varies from case to case. 

 
%an order to the database, there will be a "if order.price > 0 " to restrict the price column of orders table is always greater than zero. 

\textbf{Database constraints.}  
Many data columns are associated with constraints inside the database (DB), like the {\tt varchar(30)} constraint shown in Figure \ref{fig:crossstack}. These constraints are specified in   the applications' migration files, which are used to alter database schema over time.
%xxx\shan{a one sentence definition of migration files?}. 
The majority of them (more than 99.5\%
in our studied applications) are specified through Rails Migration APIs, and are very rarely
specified through SQL queries directly (<30 cases
across all 12 applications we checked). 
%Once specified, they can be seen in the {\tt schema.rb} file in the Rails application. 
%\alvin{what does `can be seen in the rb file' mean? Is that file automatically generated?}\junwen{Yes, it can be generated through `rake db:migrate', which will create the final db schema through accumulating the migration files.}\shan{junwen, how about
%we remove the schema.rb sentence? we never use scheme.rb in the paper, right?}\junwen{sure}

These constraints are checked by the DB when an {\tt INSERT} or {\tt /UPDATE} query to the corresponding columns is issued (either by the application or DB administrator).
%sent either by the RAILs engine or directly by any people or program having access to the DB table.
If the check fails, the application will throw an {\tt ActiveRecord::StatementInvalid}
exception to indicate an underlying DB error. 
Unfortunately, in practice, developers almost {\it never} catch such exceptions
(it is caught in only 4 cases across thousands of model object saves 
across the 12 applications we studied). Hence, once triggered,
the web user's session will most likely crash, with all the filled-in contents lost with a cryptic SQL error shown to users. 
% \shan{how to specify
% error messages and how to display error messages for these cases?}

\textbf{Why are the constraints distributed across components?}
Front-end constraints are specified for web-form input data,
which is often related to DB record (e.g., used
as query parameters, compared with query results, etc.).  
Validation functions and DB constraints are specified only for
database fields, and are checked right before 
saving data into the DB.
The expressiveness of these two are similar --- most
constraints that are expressible using validation functions can also be written using
SQL queries or migration APIs, and vice versa.
Complicated constraints expressed using custom validation functions can   be expressed in the DB layer as
{\tt CONSTRAINT CHECK}s or custom stored procedures. However, neither layer can replace the other given the existence of ``backdoors,'' e.g., DB administrators updating data using the DB console, or sharing the same DB across multiple applications. Both are common practices~\cite{discourse-import}.

%\utsav{The Rails APIs allow users to specify nearly all types of constraints. The only cases where users must use custom SQL queries are: 1) adding custom "CHECK" constraints to DB (which allow users to limit domain of values, e.g. field > 0, or create customized multi-column constraints), 2) changing primary key constraints, or 3) performing changes to format that updates/casts existing data. However this custom SQL exists in the schema.rb file (and used infrequently - 4 cases).}\shan{?? what does it mean by having these SQL queries in scheme file?}


%%%%%%%%%%%%%%COMMONT OUT%%%%%%%%%%%%
%\shan{Junwen, please revise this paragraph.
%The example used here better to match an example that we will use later}

% \cong{junwen: can you change the figure, 1) reflect the data flow or the connection between MVC; 2) add the SQL query (since you estimate the SQL query time and how it contributes to a tag, I think it is important to add a concrete query in the example). 3) add circled number to the figure such that the text can easily refer, check CIKM figure2.}


% Rails maps each model class derived from {\tt ActiveRecord} to a single database table, for example a \textbf{User} class maps to the \textbf{users} table. The 
% {\tt ActiveRecord} interface also provides APIs that will be translated into
% database queries by ORM at run time.
% how the application responds to a specific web-page request. Inside
% an action there is code to retrieve database data through queries
% transparently translated by the ORM. Finally, the retrieved data is
% rendered via views that are often written in a template language, as
% shown in index.html.erb in Figure 1. Such views determine how
% the retrieved data is displayed in a client’s browser.
% The life cycle of a Rails application, and ORM applications in
% general, is as follows. When receiving a client HTTP request like
% “http://.../messages/index”, the application server first looks
% up the routing rules, shown at the top of Figure 1, to map this
% request to the index action inside MessagesController. When the
% index action executes, it invokes the @user.undeleted_messages
% function, which calls messages. where(...). The call to the Rails
% API where is dynamically translated to a SQL query by the Rails
% framework to retrieve data from the DBMS. The query results are
% then serialized into model objects and stored in @messages. The
% index action then calls render "index" to render the retrieved
% data in @messages using the index.html.erb template.



\iffalse
% \subsection{DB-aware static analysis of Ruby on Rails}
% \label{sec:back_adg}
% During preprocessing, previous work\cong{previous, or \Tool? if previous, which work?} inlines function calls to enable inter-procedural analysis later on. This
% process involves type inference \cite{furr2009static}% xxx \shan{how?}
% , as Ruby is dynamically typed \cite{an2009static}, and special
% handling to view files and filter functions. Specifically, 
% since view files may contain computation and queries too, 
% %\Tool identifies view files rendered by each controller and appends
% %Ruby code there to the controller action. 
% \Tool extracts the ruby code in each view and inlines them to the corresponding controller where the view is rendered.
% \Tool also inlines filter functions automatically invoked before an action
% and validation functions automatically invoked before every database-modifying function. 


% Previous work~\cite{powerstation} builds the PDG for an action through the intermediate representation(IR) compiled by JRuby~\cite{jruby}. Every node $n$ in the PDG represents a statement in the JRuby IR and 
% every edge $e$ represents either control dependency or data dependency. Furthermore, a data-dependency
% edge $n_1 \rightarrow n_2$ indicates that the output object $o$ of $n_1$ is used by $n_2$ without
% other statements overwriting $o$ in between. For each statement, we keep the line number information. Combining with the APIs provided by ActiveRecord and the database schema information, we further decide whether a node is a query. 

% The PDG generated above is then extended in three ways to create the ADG: 
% (1) changing and splitting some nodes to become 
% Query nodes; (2) annotating every Query node with the database table and fields that are read or written; 
% (3) annotating every outgoing data-dependency edge of a Query node with the exact field(s) that are used.

% To accomplish this, previous work first analyzes every model class that extends the Rails
% {\tt ActiveRecord} interface to determine all the database
% tables in the application and the association relationship among them.
% For example, analyzing the model classes illustrated in 
% Figure~\ref{fig:schema}, \Tool identifies the {\tt users} table
% corresponding to the {\tt User} class and similarly for the {\tt Blog} class, and that these two models have 
% a one-to-many relationship, i.e., each instance of {\tt User} may own multiple instances of {\tt Blog}.
% Second, \Tool analyzes the {\tt schema.rb} file to determine
% how many fields each table contains. For example, parsing the
% {\tt schema.rb} snippet in Figure~\ref{fig:schema}, \Tool learns about 
% the schemas of table {\tt users} and {\tt blogs} as
% shown in the bottom of the figure.

% Third, \Tool identifies queries from three sources: (1) explicit 
% invocations of
% Rails {\tt ActiveRecord} Query APIs, such as {\tt exist?},
% {\tt reload}, \textit{\tt update}, \textit{\tt destroy}, etc;
% (2) implicit queries generated by Rails to access object fields, e.g., {\tt $o_1$.$o_2$}, where 
% the class of $o_1$ and the class of $o_2$ are associated model classes
% (e.g., {\tt user.blogs} would incur a query to
% retrieve records in {\tt blogs} table that are associated with
% the specific {\tt user} record in {\tt users} table);
% (3) explicitly invoked raw SQL queries through  Base.connection.execute.
% %\junwen{it's also thru the ActiveRecord API, similar as the first type}
% Any query identified above is represented as a Query node in the ADG.\footnote{At run time,
% multiple such queries could be composed by ORM into one SQL query.
% Such query chaining does not affect \Tool analysis.} 

\fi

\subsection{Application selection}
 There are many ORM frameworks available (e.g., Ruby on Rails, Django, Hibernate, etc.). Among them, Rails is the most popular on Github. 
Thus, we studied 12
open-source Ruby on Rails applications, including the top two most popular Ruby applications from six major categories of web applications on GitHub: Discourse (Ds) and Lobster (Lo) are forums; Gitlab (Gi) and Redmine (Re) are collaboration pplications; Spree (Sp) and Ror ecommerce (Ro) are Ecommerce applications; Fulcrum (Fu) and Tracks (Tr) are Task-management applications; Diaspora (Da) and Onebody (On) are social network applications; OpenStreetmap (OS) and FallingFruit (FF) are map applications. All of them have been actively developed for years, with hundreds to tens of hundreds of code commits. 

\subsection{Issue selection}
\begin{table} 
% \setlength{\tabcolsep}{1 pt}  
\centering 
\caption{\# of data-constraint issues in our study and the total \# of issues in the issue-tracking system}
%\footnotesize
% @{\hspace{0.1in}}r@{\hspace{0.1in}}r@{\hspace{0.1in}}r@{\hspace{0.1in}}r@{\hspace{0.1in}}r@{\hspace{0.1in}}r@{\hspace{0.1in}}r@{\hspace{0.1in}}r@{\hspace{0.1in}}r@{\hspace{0.1in}}r@{\hspace{0.1in}}r@{\hspace{0.1in}}r@{\hspace{0.1in}}
% \resizebox{0.9\columnwidth}{!}{
\begin{tabular}{lrrrrrrrrrrrr}
\toprule
 & Ds & Lo & Gi & Re & Sp & Ro & Fu & Tr & Da & On & FF & OS\\
\midrule
Studied &14 & 1 & 16 & 30 & 31 & 2 & 1 & 1 & 11 & 5 & 0 & 2\\
\midrule
Total &4607 & 220 & 18038 & 12117 & 4805 & 114 &158 &1470 &3206 & 400 & 17 & 650\\
\bottomrule
\end{tabular}
% }
\label{table:issueapp0}
% \vspace{-0.2in}
\end{table}

Section~\ref{sec:causes} studies the root causes and symptoms of real-world data constraint problems using
\numissuescon reports sampled from the above 12 applications' issue-tracking systems. For the 9 applications that have medium-size
issue databases (i.e., 100--5000 total reports), we randomly sampled 100 reports for each. For Redmine and Gitlab, which have more than 10,000 reports, we randomly sampled 200 reports for each. For FallingFruit, which only has 17 reports, we took all of them. Among the resulting 1317 sampled reports, we manually
checked all the reports that contain keywords like ``data format,'' ``data inconsistency,'' ``data constraint,'' ``format change,'' ``format conflict,'' etc. We finally obtained
\numissuescon reports that are truly related to data constraints, as shown in \tabref{issueapp0}.  

\section{Profiling Results}
\label{sec:profiling}

\begin{figure}
\centering
\includegraphics[height=1.75in, width=3.5in]{hownotto/EndToEnd}
\caption{End-to-end page loading time
}
\label{fig:eoeFig}
{\footnotesize Measured for top 10 time-consuming pages per application. Box: 25 to 75 percentile; Red line: median; PA: problematic actions from all 12 applications (see Section~\ref{sec:meth_profile}).\par}
\end{figure}


\begin{figure}
\centering
\includegraphics[height=1.75in, width=3in]{hownotto/Percentage}
\caption{Percentage of server time among end-to-end time}
{\footnotesize Measured for top 10 most time-consuming pages per application.
Red line: median; PA: problematic actions from all 12 applications (see Section~\ref{sec:meth_profile})}
\vspace{-0.2in}
\label{percentageFig}
\end{figure}

\begin{comment}
\begin{figure}
\includegraphics[height=2.2in, width=3.5in]{hownotto/probAct}
\caption{Number of problematic actions}
\label{probAct}
\end{figure}
\end{comment}
\paragraph{\bf{End-to-end loading time}} 
We identify the 10 pages with the most loading time for every application under the 20,000-record database configuration and plot their average end-to-end page loading time in
Figure~\ref{fig:eoeFig}. 11 out of 12 applications have pages whose average end-to-end loading time (i.e., from browser sending the URL request to page finishing loading) exceeds 2 seconds; 6 out of 12 applications have pages that take more than 3 seconds to load. \textit{Tracks} performs the worst: all of its top 10 most time-consuming pages take more than 2 seconds to load. 
%\shan{75\% of top 10 actions?} 
Note that, our workload is 
{\it smaller} or, for some applications, {\it much smaller} than today's real-world workload. Considering how the real-world workload's size will continue growing,
these results indicate that performance problems are prevalent and critical for deployed Rails applications.

\vspace{-0.08in} 
\paragraph{\bf{Server vs. client}}
We break down the end-to-end loading time of the top 10 pages in each 
application into server time (i.e., time for executing controller action, including view rendering and data access, on Rails server), 
client time (i.e., time for loading the DOM in the browser), and network time (i.e., time for data transfer between server and browser).  %\cong{We need to say what's the network latency. Maybe what browser client side uses also matters?} 
As shown in Figure~\ref{percentageFig}, server time contributes to at least 40\% of the end-to-end-latency for more than half of the top 10 pages in all but 1 application.\footnote{Part of the server time could overlap with the client time or the network time. However, our measurement shows that the overlap is negligible.}
Furthermore, over $50\%$ of problematic pages spend more than 80\% of the loading time on Rails server, as shown by the rightmost bar (labeled {\bf PA}) in Figure~\ref{percentageFig}.
%Furthermore, inefficiency in server computation could also affect network latency, as we will see later.
%Our profiling also indicates that server time scales much worse than client time xxx.
This result further motivates us to study the performance problems on the server side of ORM applications.
\vspace{-0.08in} 
\paragraph{\bf{Problematic server actions}}
Table~\ref{tab:probAct} shows the number of problematic actions for each application identified using
the methodology discussed in Section~\ref{sec:meth_profile}. In total, there are \numpactions problematic actions identified from the top 10 most time-consuming actions of every application. Among them, 34 have scalability problems and 28 take more than 1 second of server time. Half of the pages that correspond to these 40 problematic actions take more than 2 seconds to load, as shown in the rightmost bar (labeled {\bf PA}) in Figure~\ref{fig:eoeFig}.
In addition, we find 64 performance issues in these 40 problematic actions, and we will discuss them in detail in Section~\ref{sec:causes}.
%Server time contributes to more than half of the end-to-end time of all the problematic actions (right-most bar in Figure \ref{probAct}). \alvin{didn't we already discuss this in the last paragraph?} 
%\shan{how many have $>1$ second time?}\shan{Hmm, the figure seems to show much fewer number of actions with scalability problems}.

\begin{table}
\centering
\footnotesize
\caption{Number of problematic actions in each application}
\begin{tabular}{@{\hspace{0.1in}}l@{\hspace{0.1in}}l@{\hspace{0.1in}}l@{\hspace{0.1in}}l@{\hspace{0.1in}}l@{\hspace{0.1in}}l@{\hspace{0.1in}}l@{\hspace{0.1in}}l@{\hspace{0.1in}}l@{\hspace{0.1in}}l@{\hspace{0.1in}}l@{\hspace{0.1in}}l@{\hspace{0.1in}}l@{\hspace{0.1in}}}
\toprule
App & Ds & Lo & Gi & Re & Sp & Ro & Fu & Tr & Da & On & FF & OS  \\
\midrule
slow & 0 & 0 & 1 & 1 & 3 & 0 & 0 & 0 & 0 & 1 & 0 & 0 \\
not-scalable & 1 & 1 & 0 & 0 & 0 & 0 & 2 & 0 & 1 & 2 & 3 & 1 \\
slow \& not-scalable & 0 & 5 & 1 & 2 & 0 & 2 & 1 & 10 & 1 & 0 & 1 & 0\\
 \bottomrule
\end{tabular}
\label{tab:probAct}
\vspace{-0.25in}
\end{table}


\section{Causes of Inefficiencies}
\label{sec:causes}


After studying the 64 performance issues in the \numpactions problematic actions and the \numissues issues reported in the applications' bug-tracking systems, we categorize the inefficiency causes into three categories: ORM API misuses, database design, and application design. In the following we discuss these causes and how developers have addressed them. We believe these causes apply to applications built using other ORM frameworks as well, as we will discuss in Section~\ref{sec:dis}. 
%The performance causes that we have observed from performance issue tracking systems and performance action profiling are largely consistent. Also, the way how developers fix the performance issues from issue tracking systems are consistent with what we have done for the profiling issues. Consequently, we will discuss them together below. 

%\shan{are these ALL inefficiency problems or ALL inefficient-data-processing problems?}\shan{What about rendering problems; rendering template optimization?}

\begin{comment}

\end{}

\newcolumntype{g}{>{\columncolor{Gray}}r}
\begin{table*}[]
\centering
\small
\caption{Causes for problematic actions and performance issues}
\label{tab:freq}
\begin{tabular}{@{}llggggggggggggg@{}}
\toprule

\rowcolor{white}
 & Causes &Ds & Lo & Gi & Re & Sp & Ro & Fu & Tr & Da & On & FF & OS& SUM  \\
 \midrule
\rowcolor{white}
 & Inefficient  & { 0} & {0} & 0 & 0 & 0 & 1 & 0 & 1 & 2 & 2 & 2 & 0 & 8 \\
\cmidrule{3-15}
 & computation & 0 & 0 & 3 & 6 & 5 & 0 & 0 & 2 & 2 & 0 & 0 & 0 & 18 \\ 
\cmidrule{2-15}

\rowcolor{white}
 & Unnecessary  & 0 & 3 & 0 & 0 & 0 & 0 & 0 & 0 & 0 & 0 & 2 & 0 & 5 \\
\cmidrule{3-15}
API & Computation & 1 & 0 & 3 & 4 & 4 & 1 & 0 & 1 & 2 & 1 & 0 & 0 & 17 \\
\cmidrule{2-15}

\rowcolor{white}
Usage & Inefficient  & 0 & 1 & 0 & 0 & 3 & 2 & 0 & 2 & 2 & 3 & 0 & 1 & 14 \\
\cmidrule{3-15}
 &data accessing  & 3 & 0 & 4 & 5 & 10 & 0 & 0 & 2 & 0 & 2 & 0 & 0 & 26 \\
\cmidrule{2-15}
\rowcolor{white}
 & Unnecessary & 0 & 0 & 1 & 0 & 0 & 0 & 0 & 0 & 0 & 0 & 0 & 0 & 1 \\
\cmidrule{3-15}
 & data loading  & 2 & 0 & 3 & 1 & 2 & 0 & 0 & 0 & 0 & 0 & 0 & 0 & 8 \\
\cmidrule{2-15}
 \rowcolor{white}
 & Inefficient Rendering & 0 & 3 & 1 & 0 & 0 & 0 & 0 & 1 & 0 & 0 & 0 & 0 & 5 \\
\midrule


\rowcolor{white}
 & Missing  & 0 & 0 & 0 & 1 & 0 & 0 & 0 & 0 & 0 & 0 & 1 & 1 & 3 \\
\cmidrule{3-15}
 & Fields & 0 & 2 & 0 & 0 & 2 & 0 & 0 & 0 & 0 & 0 & 1 & 0 & 5 \\
\cmidrule{2-15}

\rowcolor{white}
Database& Missing  & 0 & 0 & 0 & 0 & 0 & 0 & 0 & 1 & 0 & 0 & 0 & 0 & 1 \\
\cmidrule{3-15}
 Design& association & 0 & 1 & 0 & 0 & 1 & 0 & 0 & 0 & 1 & 0 & 0 & 0 & 3 \\
 
\cmidrule{2-15}

\rowcolor{white}
   & Missing  & 0 & 1 & 0 & 0 & 0 & 0 & 0 & 0 & 0 & 0 & 2 & 0 & 3 \\
\cmidrule{3-15}
 & index & 3 & 1 & 4 & 6 & 3 & 0 & 0 & 3 & 5 & 1 & 1 & 3 & 30 \\
\midrule

\rowcolor{white}
& Unworthy  & 1 & 0 & 0 & 2 & 0 & 2 & 6 & 10 & 0 & 1 & 0 & 0 & 22 \\
\cmidrule{3-15}
App & content & 5 & 1 & 1 & 0 & 0 & 1 & 0 & 3 & 1 & 0 & 0 & 2 & 14 \\
 \cmidrule{2-15}
\rowcolor{white}
Design & Unworthy  & 0 & 2 & 0 & 0 & 0 & 0 & 0 & 0 & 0 & 0 & 0 & 0 & 2 \\
 \cmidrule{3-15}
 & features & 3 & 2 & 4 & 0 & 1 & 0 & 2 & 1 & 2 & 1 & 1 & 2 & 19 \\
 \midrule
 

\rowcolor{white}
SUM &  & 18 & 17 & 24 & 25 & 31 & 7 & 8 & 27 & 17 & 11 & 10 & 9 & 204\\
\bottomrule
\end{tabular}
\\
\footnotesize{data with white background is for problematic actions from 12 representative applications\\ data with gray background is for performance issues from 12 bug-tracking systems}
\end{table*}
\end{comment}
\definecolor{LightCyan}{rgb}{0.88,1,1}
%\definecolor{Gray}{gray}{0.9}
\definecolor{Gray}{rgb}{0.6,0.6,0.6}
\newcolumntype{g}{>{\columncolor{Gray}}r}
\begin{table}[]
\centering
\small
\caption{Inefficiency causes across 12 applications}
\label{tab:freq}
\begin{tabular}{@{\hspace{0.1in}}c@{\hspace{0.1in}}g@{\hspace{0.1in}}g@{\hspace{0.1in}}g@{\hspace{0.1in}}g@{\hspace{0.1in}}g@{\hspace{0.1in}}g@{\hspace{0.1in}}g@{\hspace{0.1in}}g@{\hspace{0.1in}}g@{\hspace{0.1in}}g@{\hspace{0.1in}}g@{\hspace{0.1in}}g@{\hspace{0.1in}}g@{\hspace{0.1in}}}
\toprule

\rowcolor{white}
   &Ds & Lo & Gi & Re & Sp & Ro & Fu & Tr & Da & On & FF & OS& Sum  \\
 \midrule
\rowcolor{white}
\multicolumn{13}{c}{\bf ORM API Misuse}\\
\midrule
\rowcolor{white}
\multirow{ 2}{*}{IC}  & { 0} & {0} & 0 & 0 & 0 & 1 & 0 & 1 & 2 & 2 & 2 & 0 & 8 \\
\cmidrule{2-14}

  & 0 & 0 & 3 & 6 & 5 & 0 & 0 & 2 & 2 & 0 & 0 & 0 & 18 \\ 
\cmidrule{1-14}

\rowcolor{white}
\multirow{ 2}{*}{UC}   & 0 & 3 & 0 & 0 & 0 & 0 & 0 & 0 & 0 & 0 & 2 & 0 & 5 \\
\cmidrule{2-14}
  & 1 & 0 & 3 & 4 & 4 & 1 & 0 & 1 & 2 & 1 & 0 & 0 & 17 \\
\cmidrule{1-14}

\rowcolor{white}
 \multirow{ 2}{*}{ID}   & 0 & 1 & 0 & 0 & 3 & 2 & 0 & 3 & 2 & 3 & 0 & 1 & 15 \\
\cmidrule{2-14}
   & 3 & 1 & 4 & 5 & 11 & 0 & 0 & 2 & 1 & 2 & 0 & 0 & 29 \\
\cmidrule{1-14}
\rowcolor{white}
  \multirow{ 2}{*}{UD}  & 0 & 0 & 1 & 0 & 0 & 0 & 0 & 0 & 0 & 0 & 0 & 0 & 1 \\
\cmidrule{2-14}
    & 2 & 0 & 3 & 1 & 2 & 0 & 0 & 0 & 0 & 0 & 0 & 0 & 8 \\
\cmidrule{1-14}
 \rowcolor{white}
   IR & 0 & 3 & 1 & 0 & 0 & 0 & 0 & 1 & 0 & 0 & 0 & 0 & 5 \\
\midrule

\rowcolor{white}
\multicolumn{13}{c}{\bf Database Design Problems}\\
\midrule
\rowcolor{white}
  \multirow{ 2}{*}{MF}   & 0 & 0 & 0 & 1 & 0 & 0 & 0 & 0 & 0 & 0 & 1 & 1 & 3 \\
\cmidrule{2-14}
   & 0 & 2 & 0 & 0 & 2 & 0 & 0 & 0 & 0 & 0 & 1 & 0 & 5 \\
\cmidrule{1-14}
\rowcolor{white}
   \multirow{ 2}{*}{MI}   & 0 & 1 & 0 & 0 & 0 & 0 & 0 & 0 & 0 & 0 & 2 & 0 & 3 \\
\cmidrule{2-14}
   & 3 & 1 & 4 & 6 & 3 & 0 & 0 & 3 & 5 & 1 & 1 & 3 & 30 \\
\midrule
\rowcolor{white}
\multicolumn{13}{c}{\bf Application Design Tradeoffs}\\
\midrule
\rowcolor{white}
 \multirow{ 2}{*}{DT}   & 1 & 0 & 0 & 2 & 0 & 2 & 6 & 10 & 0 & 1 & 0 & 0 & 22 \\
\cmidrule{2-14}
  & 5 & 1 & 1 & 0 & 0 & 1 & 0 & 3 & 1 & 0 & 0 & 2 & 14 \\
 \cmidrule{1-14}
\rowcolor{white}
  \multirow{ 2}{*}{FT}   & 0 & 2 & 0 & 0 & 0 & 0 & 0 & 0 & 0 & 0 & 0 & 0 & 2 \\
 \cmidrule{2-14}
  & 3 & 2 & 4 & 0 & 1 & 0 & 2 & 1 & 2 & 1 & 1 & 2 & 19 \\
 \midrule
 

\rowcolor{white}
Sum  & 18 & 17 & 24 & 25 & 31 & 7 & 8 & 27 & 17 & 11 & 10 & 9 & 204\\
\bottomrule
\end{tabular}
\\
\footnotesize{Data with white background shows 64 issues from 40 problematic actions\\ Data with gray background shows 140 issues from 12 bug-tracking systems\\
\begin{tabular}{rlrl}
IC:& Inefficient Computation & MF:& Missing Fields \\
UC:& Unnecessary Computation & MI:& Missing Indexes \\
ID:& Inefficient Data Accessing & DT:& Content Display Trade-offs\\
UD:& Unnecessary Data Retrieval & FT:& Functionality Trade-offs\\
IR:& Inefficient Rendering & \\

\end{tabular}
}

\vspace{-0.10in}
\end{table}

\subsection{ORM API Misuses} 
\label{causes:api}

About half of the performance issues that we studied suffer from API misuses. In these cases, performance can be improved by changing how the Rails APIs are used without modifying program semantics or database design. While some of these misuses appear simple, making the correct decision requires deep expertise in the implementation of the ORM APIs and query processing.

%\shan{TODO: 1.discuss scalability problems 2. add some profiling examples, instead of all examples from bug-tracking system?}
\vspace{-0.1in}

\subsubsection{Inefficient Computation (IC)}
\label{sec:inefficomp}

In these cases, the poorly performing code conducts useful computation but inefficiently.
Such cases comprise more than 10\% of the performance issues in both bug reports and problematic actions. 
\vspace{-0.18in} 
\paragraph{\bf{Inefficient queries}}
The same operation on persistent data can be implemented via different ORM calls. However, the performance of the generated queries can be drastically different. This problem has not been well studied before for ORM applications.
%A functionality can be accomplished by different database queries and developers unfortunately use a more expensive query. This problem is widespread and is difficult to avoid for developers who cannot easily track which queries are transparently generated by ORM or whether the queries are efficient. Unfortunately, there was no automated solution yet.

Figure~\ref{fig:spreeAnyVsExists} shows two ways that an online shopping system checks if there are product \texttt{variants} whose inventory are not tracked.
%(i.e., \texttt{track\_inventory} field is \texttt{false}).
The Ruby code differs only in the use of 
\texttt{any?} vs \texttt{exists?}. However, the performance of the generated queries differs substantially:
the generated query in Figure~\ref{fig:spreeAny} scans all records in the \texttt{variants} table to compute the count if no index exists, but that in Figure~\ref{fig:spreeExists} only needs to scan and locate the first 
\texttt{variant} record where the predicate evaluates to true. \texttt{Spree} developers discovered and fixed this problem in 
\texttt{Spree-6720}.\footnote{We use \texttt{A-n} to denote report number {\tt n} in application {\tt A}'s bug-tracking system.}
% we mention that this is widespread in 2 sentences
%This problem is widespread as we find that it contributes to 3 problematic actions in 2 applications. 
Our profiling finds similar problems. For example, simply replacing \texttt{any?} with \texttt{exists?} in a problematic action of \texttt{OneBody} improves server time by 1.7$\times$. 
%\alvin{how about say how much improvement for the spree code shown rather than citing another example?}
Our static checker that will be discussed in Section~\ref{sec:dis} finds that this is a common problem as it appears in the latest versions of 9 out of 12 applications under study.

\begin{comment}
\begin{figure}[h]
  \centering
  \begin{subfigure}[t]{0.5\textwidth}
  \includegraphics[width=0.95\columnwidth]{figs/countstar}
  %\caption{Inefficient API}
  \label{InefficientAPI}
  \end{subfigure}
  
    \centering
  \begin{subfigure}[t]{0.5\textwidth}
  \includegraphics[width=0.95\columnwidth]{figs/exist}
  	%\caption{Efficient API}
  \label{efficientAPI}
  \end{subfigure}

  \caption{Inefficient query due to improper APIs in Redmine}
  \label{API}
\end{figure}
\end{comment}

\begin{comment}
\begin{figure}[h]
  \centering
  \begin{subfigure}[t]{0.5\textwidth}
  \includegraphics[width=0.95\columnwidth]{figs/any}
  \caption{Inefficient API}
  
  \label{fig:any}
  \end{subfigure}
  
    \centering
  \begin{subfigure}[t]{0.5\textwidth}
  \includegraphics[width=0.95\columnwidth]{figs/exists}
  	\caption{Efficient API}
  \label{fig:exists}
  \end{subfigure}

  \caption{Inefficient query due to improper APIs in Onebody}
  \label{fig:anyVsExists}
\end{figure}
\end{comment}

\begin{figure}
  \vspace{-0.05in}
  \centering
  \begin{subfigure}[t]{60mm}

\includegraphics[width=0.95\columnwidth]{figs/spreeAny}
  \caption{Inefficient}
  
  \label{fig:spreeAny}
  \end{subfigure}
  
    \centering
  \begin{subfigure}[t]{60mm}
  \includegraphics[width=0.95\columnwidth]{figs/spreeExists}
  	\caption{Efficient}
    
  \label{fig:spreeExists}
  \end{subfigure}

  \caption{Different APIs cause huge performance difference}
  %Inefficient query due to improper APIs in Spree}
  \label{fig:spreeAnyVsExists}
  \vspace{-0.25in}
\end{figure}

Another common problem is developers using API calls that generate queries with unnecessary ordering of the results. For example, Ror, Diaspora, and Spree developers use \texttt{Object.}\texttt{where(c).first} to get an object satisfying predicate \texttt{c} instead of \texttt{Object.find\_by(c)}, not realizing that the former API orders {\tt Object}s by primary key after evaluating predicate {\tt c}. 
As a fix, both Gitlab and Tracks developers explicitly add \texttt{except(:order)} in 
the patches to eliminate unnecessary ordering in the queries, further showing how simple changes can lead to drastic performance difference.

\vspace{-0.08in} 
\paragraph{\bf{Moving computation to the DBMS}} %Should Be a Query}}
As the ORM framework hides the details of query generation, developers often write code that results in multiple queries being generated. Doing so incurs extra network round-trips, or running computation on the server rather than the DBMS, which leads to performance inefficiencies.

%
%A functionality is faster to accomplish by one query. Unfortunately, the developers use either (1) multiple queries or (2) one query together with in-memory computation. Either way, inefficiency is incurred by extra query and/or data round-trips. Although the general problem of pushing application computation down to database has been tackled by previous work \cite{cheung:pldi13} using program synthesis, our study finds that many such problems in Rails are caused by simple API misuses and hence could benefit from a new tool that is less general but much simpler.
%
%There are many cases that only involve one expression and simple API misuses.
For example, the patch of \texttt{Spree-6720} replaces
\texttt{if(exist?)} \texttt{find; else create} with 
\texttt{find\_or\_create\_by}, where the latter combines two queries that are issued by \texttt{exist}
and \texttt{find}/\texttt{create} into one.
The patch of \texttt{Spree-6950} replaces
\texttt{pluck(:total).sum} with \texttt{sum(:total)}. The former uses \texttt{pluck} to issue
a query to load the \texttt{total} column of all corresponding records and then
computes the sum in memory, while the latter uses \texttt{sum} to issue a 
query that directly performs the sum in the DBMS without returning actual records to the server.
The patch of {\tt Gitlab-3325} replaces \texttt{pluck(:id)+pluck(:id)}, which replaces two queries and an in-memory union via {\tt +} with one SQL \texttt{UNION} query, in effect 
moving the computation to the DBMS.
%conducts the summation in database and hence saves data loading time. 
%This was from the bug-tracking system of Gitlab, and we also see similar cases in Spree.
Such API misuses are very common and occur in many applications as we 
will discuss in Section~\ref{sec:dis}. 



%For example, xxx developers use \texttt{array.count} instead of \texttt{array.size} to get the size of an array. When \texttt{array} is already loaded in memory, the former issues a query to do the counting in database, while the latter conducts in-memory size computation and hence is much more efficient.

There are also more complicated cases where a loop implemented in Ruby can be completely pushed
down to DBMS,  which has been addressed in
previous work using program synthesis~\cite{cheung:pldi13}.

%where the Rails program iterate through a container
%filled with data previously loaded from database 
%As another example, it could cause many rows to be loaded to memory for selection that should have been done in database.
%An example of row-wise unnecessary data retrieval is shown in Fig \ref{rowwise}. Ruby code in Fig \ref{posts} from \texttt{Spree-6903} first retrieves all posts of ``StatusMessage'' type, and then through \texttt{tagged\_with}, only posts with specified \texttt{tagged\_name} will be rendered. Other posts are useless. A better way to avoid unnecessary data is to issue another query which will only retrieve the posts with specified \texttt{tag\_name} as shown in Fig \ref{taggings}. This simple change would save the corresponding action in software xxx xx based on our profiling.

%Redundant row retrieval problem has not been studied by previous work.
%\shan{is this right?} 

%Furthermore, it is difficult for developers to avoid this problem. In fact, the more efficient code shown in Figure \ref{rowwise} may actually be considered as ``message chain'' code smell. It is difficult for developers to realize that such smell code is actually much more efficient. Future work should build xx to automatically detect and fix this xxx.

%\begin{figure}[h]
%  \centering
%  \begin{subfigure}[t]{0.5\textwidth}
%  \includegraphics[width=1\columnwidth]{figs/posts}
%  \caption{Unoptimized} 
%  \label{posts}
%  \end{subfigure} 
%    \centering
%  \begin{subfigure}[t]{0.5\textwidth}
%  \includegraphics[width=1\columnwidth]{figs/taggings}
%  	\caption{Optimized}
%  \label{taggings}
%  \end{subfigure}
%  \caption{Unnecessary row-data retrieval in xxx}
%  \label{rowwise}
%\end{figure}
\vspace{-0.08in} 
\paragraph{\bf{Moving computation to the server}} %Should Not be a Query}} 
Interestingly, there are cases where the computation should be moved to the server from the DBMS. %a functionality is faster to do by in-memory computation instead of a database query. 
As far as we know, this issue has not been studied before. 
%and is naturally out of the scope of database optimization. 

%
For example, in the patch of \texttt{Spree-6819}, developers replace 
\texttt{Objects.count} with \texttt{Objects.size} in 17 different locations, as
\texttt{count} always issues a \texttt{COUNT} query while \texttt{size} counts the 
\texttt{Objects} in memory
if they have already been retrieved from the database by earlier computation.
%and issues a count SQL query otherwise. 
%However, as all instances are in views (not controllers) or in utility functions, merging them with other code such that the appropriate call can be made will break code modularity, making it a difficult choice for developers.
%Consequently, code modularity concerns prevent them from being merged with other queries. 
Such issues are also reported in {\tt Gitlab-17960}.

\vspace{-0.08in} 
\paragraph{\bf{Summary}} Rails, like other ORM frameworks, lets developers implement a given functionality in various ways.
%With ORM framework, a functionality can often be implemented by different combinations of queries and in-memory computation. 
Unfortunately, developers often struggle at 
picking the most efficient option. The deceptive names of many 
Rails APIs like \texttt{count} and \texttt{size} make this even more challenging. Yet, we believe many cases can be fixed using simple static analyzers, as we will discuss in Section~\ref{sec:dis}.
%Meanwhile, many inefficient API patterns are simple and repeatedly appear in different applications. It is both feasible and very helpful to build static analysis tools to automatically identify and/or fix inefficient API uses. More complicated cases would benefit from general query synthesis techniques \cite{cheung:pldi13}.

\begin{figure}
  \centering
  \includegraphics[width=0.7\columnwidth]{hownotto/redCom}
  \caption{A loop-invariant query in Redmine}
  \label{fig:redCom}
\end{figure}

\subsubsection{Unnecessary Computation (UC)} 
\label{sec:uncomp}
More than 10\% of the performance issues are caused by (mis)using ORM APIs that lead to unnecessary queries being issued. This type of problems has not been studied before.
%The poorly performing code issues unnecessary queries. It happens for about 10\% of studied performance issues, and also causes problematic actions in the latest versions of multiple applications.
\vspace{-0.08in} 
\paragraph{\bf{Loop-invariant queries}}
Sometimes, queries are repeatedly issued to load the {\it same} database contents and hence are unnecessary. 
%In traditional programs, these problems can be solved by classic optimization techniques, which we will discuss below. Unfortunately, problems here require the optimizer to understand both ORM/query semantics and Ruby application semantics, and have not been tackled by previous research.
%
%Sometimes, a query is repeatedly issued in a loop to return the same content --- it could be optimized by \textit{loop-invariant code motion} if the optimizer understands not only Ruby logic but also ORM and queries.
For instance, Figure~\ref{fig:redCom} shows the patch from \texttt{redmine-23334}.
This code iterates through every custom field 
\texttt{value} %that is visible to the \texttt{user} 
and retains only those that 
\texttt{user} has write access to.
%, \texttt{reject}ing the read-only ones. 
To conduct this access-permission checking, in every iteration, \texttt{read\_only\_attribute\_}
\texttt{names(user)} issues a query to get the names of all read-only fields of \texttt{user}, as shown by the red highlighted line
in the figure. Then, if
\texttt{value} belongs to this read-only set, it will be excluded from the return set of this function (i.e., the \texttt{reject} at the beginning of the loop takes effect). Here, the \texttt{read\_only\_attribute\_names(user)} query 
returns exactly the same result during every iteration of the loop and causes
unnecessary slowdowns. 
%\alvin{I am confused. What is the original code? The red line? Why are we showing the a patch but not discussing it. Add line numbers?}
As shown by the green lines in figure, Redmine developers hoist loop invariant \texttt{read\_only\_attribute\_names(user)} outside the loop and achieve more than 20$\times$ speedup for the corresponding function for their workload.
%, shortening its time from 1.03 second to 0.05 second with 1000 issues.
Similar issues also occur in Spree and Discourse.





\vspace{-0.08in} 
\paragraph{\bf{Dead-store queries}}
In such cases, queries are repeatedly issued to load {\it different} database contents into the same memory object while the object has not been used between the reloads.
%without using the object in between --- it could be optimized by
%\textit{dead-store elimination} if the optimizer understands not only Ruby but also ORM and queries. 
 For example, in Spree, every shopping transaction has a corresponding
 {\tt order} record in the {\tt orders} table. This table has a
 {\tt has\_many} association relationship with the 
 {\tt line\_items} table, meaning that every order  contains
 multiple lines of items. Whenever the user updates his/her shopping
 cart, the {\tt line\_items} table would change, at which point the old version of Spree always uses an {\tt order.reload} to make 
 sure that the in-memory copy of {\tt order} and its associated 
 {\tt line\_item}s are up-to-date. Later on, developers realize that
 this repeated reload is unnecessary, because the content of 
 {\tt order} is not used by the program until check out.
 Consequently, in {\tt Spree-6379}, developers remove many 
 {\tt order.reload} from model classes, and instead add it in a few places in the {\tt before\_payment} action of the
 {\tt checkout} controller, where the {\tt order} object is to be used.
%\alvin{how does this justify issuing repeated reads?} 


%original listing
\begin{comment}

\begin{lstlisting}[language=Ruby, numbers=left, firstnumber=1, numberstyle=\tiny\color{gray}, caption={redundant computation from redmine~\cite{redmine}.},label={redundantComputation}, numbers = none]
@\lstlabel{redundantComputation}@ visible_custom_field_values(user).reject do |value| 
 read_only_attribute_names(user).include?(value.custom_field_id.to_s) 
@\lstlabel{end}@ end
\end{lstlisting}

\begin{lstlisting}[language=Ruby, numbers=left, firstnumber=1, numberstyle=\tiny\color{gray}, caption={remove redundant computation from redmine~\cite{redmine}.},label={removeredundantComputation}, numbers = none]
@\lstlabel{removeredundantComputation}@ 
read_only_attr_names_array = read_only_attribute_names(user)
visible_custom_field_values(user).reject do |value|
  read_only_attr_names_array.include?(value.custom_field_id.to_s)
end
\end{lstlisting}

\end{comment}

\begin{figure}

  \centering
  \includegraphics[width=0.7\columnwidth]{hownotto/tracks63}
  \caption{A query with known results in Tracks}
  \label{fig:tracks63}
\end{figure}

\vspace{-0.08in} 
\paragraph{\bf{Queries with known results}}
A number of issues are due to issuing queries whose results are already known, hence incurring unnecessary network round trips and query processing time.
%Occasionally, certain value of a Rails API parameter would cause the corresponding query to return a predictable constant result. If that specific parameter value is common, it is better to issue the query conditionally. This optimization again requires knowledge about both application semantics and ORM/query, and is not handled by existing techniques.
%
An example is in \texttt{Tracks-63}. As shown in Figure~\ref{fig:tracks63},
the code originally issues a query to retrieve up to 
\texttt{show\_number\_completed} number of completed tasks. Clearly, when \texttt{show\_number\_completed} is $0$, the query always returns an empty set due to {\tt limit} being $0$.
Developers later realize
that $0$ is a very common setting for \texttt{show\_number\_completed}. 
%In fact, the corresponding view file \texttt{show.html.erb} has already been designed to render the completed-todo panel only conditionally.
Consequently, they applied the patch shown in Figure \ref{fig:tracks63} to only issue the query when needed. 
%\alvin{say something about how prevalent this is?}

\vspace{-0.08in} 
\paragraph{\bf{Summary}}
While similar issues in general purpose programs can be eliminated using classic compiler optimization techniques (e.g., loop invariant motion, dead-store elimination), doing so for ORM applications is difficult as it involves understanding database queries.
% and performing inter-action data-flow analysis. 
We are unaware of any compilers that perform such transformations.
%Correctly detecting and fixing unnecessary computation discussed above are non-trivial. 
%Furthermore, they often require inter-action data-flow analysis. 
%ORM developers will benefit from techniques that integrate ORM and database knowledge into traditional compiler optimization techniques like loop-invariant code motion, dead-store elimination, and others.

\vspace{-0.08in} 
\subsubsection{Inefficient Data Accessing (ID)} 
\label{sec:iffidata}
Problems under this category suffer from data transfer slow downs, including not batching data transfers (e.g., the well-known ``N+1'' problem) or batching too much data into one transfer.

\vspace{-0.08in} 
\paragraph{\bf{Inefficient lazy loading}}
As discussed in Section~\ref{sec:background}, when a set of objects $O$ in table $T_1$ are requested, objects stored in table $T_2$ associated with $T_1$ and $O$ can be loaded together through eager loading. If lazy loading is chosen instead, 
one query will be issued to load $N$ objects from $T_1$, and then
$N$ separate queries have to be issued to load associations of each such object from $T_2$. This is known as the ``N+1'' query problem. While prior work has studied this problem~\cite{nplusone, cheung:sigmod14:sloth, bullet}, we find it still prevalent: it appears in 15 problematic actions and 9 performance issues in our study. 
%This is a well known problem and can be tackled by research techniques \cite{} and Rails plugins \cite{bullet}.

Figure~\ref{fig:nplusone} shows an example that we find in the latest version of Lobsters, where the deleted code retrieves 50 \texttt{mods} objects. Then, for each \texttt{mod}, a query is issued to retrieve its associated \texttt{story}. Using eager loading in the added line, all 51 queries (and hence 51 network round-trips) will be combined together. In our experiments, the optimization reduces the end-to-end loading time of the corresponding page from 1.10 seconds to 0.34 seconds.

% Bullet is a rails gem, Sloth
%Several previous work can help developers avoid the above lazy loading problems. For example, a Rails plugin, Bullet\cite{bullet}, can warn developers about ``N + 1'' problems and remind them to use eager loading. Sloth \cite{cheung:sigmod14:sloth} will automatically batch queries to avoid N + 1 queries without developers' specifying loading strategies
%\shan{hmm, is this lazy loading or eager loading}.


\begin{figure}
  \centering
  \includegraphics[width=0.7\columnwidth]{hownotto/moderations}
  \caption{Inefficient lazy loading in Lobsters}
  \label{fig:nplusone}
\end{figure}

\begin{comment}
\begin{lstlisting}[language=Ruby, numbers=left, firstnumber=1, numberstyle=\tiny\color{gray}, caption={N + 1 queries from redmine~\cite{redmine}.},label={nplusone}]
mods = Moderation.order("id desc").limit(50)
mods.each do |mod|
  if mod.user
  	puts mod.user.username
  end
end
\end{lstlisting}

\begin{lstlisting}[language=Ruby, numbers=left, firstnumber=1, numberstyle=\tiny\color{gray}, caption={remove N + 1 queries from lobsters~\cite{lobsters}\shan{redmine?}.},label={rmnplusone}, numbers = none]
mods = Moderation.includes(:user).order("id desc").limit(50)
\end{lstlisting}
\end{comment}

\vspace{-0.08in} 
\paragraph{\bf{Inefficient eager loading}}
However, always loading data eagerly can also cause problems. \textcolor{black}{ Specifically, when the associated objects are too large, loading them all at once will create huge memory pressure and even make the application unresponsive.}
In contrast to the ``N+1'' lazy loading problem, there is little support for developers to detect eager loading problems. %\junwen{it seems in the profiling result all about inefficient lazy loading}

In \texttt{Spree-5063}, a Spree user complains that their installation performs very poorly on the product search page. Developers found that the problem was due to eager loading shown in Figure~\ref{fig:spree5063}.
In the user's workload, while loading 405 \texttt{products} to display on the page, eager loading causes 13811 related \texttt{variants} products containing 276220 \texttt{option\_values} (i.e., product information data) to be
loaded altogether, making the page freeze. As shown in 
Figure \ref{fig:spree5063}, the patch delays the loading of
\texttt{option\_values} fields of \texttt{variants} products. Note that these
\texttt{option\_values} are needed by later computation, and the patch
delays but not eliminates their loading.

\begin{figure}
  \centering
\includegraphics[width=0.7\columnwidth]{hownotto/spree5063}
  \caption{Inefficient eager loading in Spree}
  \label{fig:spree5063}
  \vspace{-0.25in}
\end{figure}

%When loadING a \texttt{product}, the corresponding \texttt{variants} and its \texttt{option\_values} and \texttt{option\_types} will be retrieved together. It is complained that it can cause the ruby process to lock up on large data sets (405 \texttt{products}/13811 \texttt{variants}/48 \texttt{optoin\_types}/948 \texttt{option\_values}). Since a variant can have many \texttt{option\_values}, the variant on average could have 20 \texttt{option\_values}. In total, there could be 276220 \texttt{option\_values} to load.  %Furthermore, ``eager loading'' may retrieve data which will never be used by the application,
%\shan{is it true?}.
%\cong{N+1 is a well-known problem. I think we may need to go deeper into the cause. As for lazy loading, why would developers load more data than needed? (programmability or modularity?)} 

\vspace{-0.08in} 
\paragraph{\bf{Inefficient updating}}
Like the ``N+1'' problem, developers would issue N queries to update N records separately (e.g., \texttt{objects.each |o| o.update end}) rather than merging them into one update (e.g., \texttt{objects.update\_all}). This is reported in Redmine and Spree, and our static checker (to be discussed in Section~\ref{sec:dis}) finds this to be common in the latest versions of 6 out of the 12 studied applications. 
\vspace{-0.22in} 
\subsubsection{Unnecessary Data Retrieval (UD)} 
%Since ORM frameworks are not aware of the high level application semantics. They cannot figure out how developers will use the data returned from the DBMS. Thus,  providing optimal data retrieval approach is not an easy job.
Unnecessary data retrieval happens when software retrieves persistent data that is not used later. Prior work has identified this problem in applications built using both Hibernate~\cite{chen:se16:redundantData} and Rails~\cite{yan:cikm17}. In our study, we find this continues to be a problem in one problematic action in the latest version of Gitlab and 9 performance issue reports.
Particularly, fixing the unnecessary data retrieval in the latest 
version of Gitlab can drop the end-to-end loading time of its 
\texttt{Dashboard/Milestones/index} page from 3.0 to 1.1 seconds in our experiments.
We also see some unnecessary data retrieval caused by simple misuses
of APIs that have similar names --- \texttt{map(\&:id)} retrieves the whole
record and then returns the \texttt{id} field, yet \texttt{pluck(:id)} only
retrieves the \texttt{id} field. 
%TODO \alvin{This is a weak point given (our own) prior work and also we are not showing code. I suggest removing it if we need space.}
%there are 2 cases caused by map vs pluck
\vspace{-0.08in} 
\subsubsection{Inefficient Rendering (IR)}
\label{sec:iffirender}
IR reflects a trade-off between readability and performance when a view file renders a set of objects. It has not been studied before.

Given a list of objects to render, developers often 
%implement a function, or use an existing library function
use a library function, like \texttt{link\_to} on Line 4 of Figure~\ref{fig:partialA},
to render one object and encapsulate it in a partial view file
such as \texttt{\_milestone.html.haml} in Figure~\ref{fig:partialA}.
Then, the main view file \texttt{index.html.haml} simply applies the
partial view file repeatedly to render all objects. 
The inefficiency is that a rendering function like \texttt{link\_to}
is repeatedly invoked to generate very similar HTML code. Instead,
the view file could generate the HTML code for one object,
and then use simple string substitution, 
such as \texttt{gsub} in Figure~\ref{fig:partialB}, to quickly 
generate the HTML code for the remaining objects, avoiding redundant
computation. The latter way of
rendering degrades code readability, but improves performance substantially
when there are many objects to render or with complex rendering functions.

Although slow rendering is complained, such transformation has not yet been proposed by issue reports. Our profiling finds such optimization speeds up 5 problematic actions by 2.5$\times$ on average.
%This type of problems does not exist in the 140 issue reports that we studied. 
%However, our profiling finds the above transformation to speed up 5 problematic actions by 2.5$\times$ on average. 

\begin{figure}
\centering
\label{fig:sl}
    \begin{subfigure}
    \centering
        \includegraphics[width=0.4\linewidth]{hownotto/partialA.pdf}
       \caption{Inefficient partial rendering}
         \label{fig:partialA}
    \end{subfigure}
    \begin{subfigure}
        \centering
        \includegraphics[width=0.4\linewidth]{hownotto/partialB.pdf}
       \caption{Efficient partial rendering}
        \label{fig:partialB} 
    \end{subfigure}
\caption{Inefficient partial rendering in Gitlab}
\end{figure}

\subsection{Database Design Problems}
\label{causes:db}

Another important cause of performance problems is suboptimal database design. Fixing it requires changing the database schema.
%Sometimes, the root cause of a performance problem is about database table design. Consequently, fixing these problems requires modifying schema and model files in ORM applications.
\vspace{-0.08in} 
\subsubsection{Missing Fields (MF)}
\label{sec:addfield}
%ORM frameworks encourage developers to maintain all classes to be stored in the database as model classes \alvin{what does Rails call such classes?} \alvin{find something to cite if possible regarding standard practices} for modularity and code maintenance. 
Deciding which object field to be physically stored in database is a non-trivial part of database schema design. %Similarly, which fields should be persistent in a model class is not always clear. 
If a field can be easily derived from other fields, storing it in database may waste storage space and I/O time when loading an object; if it is expensive to compute, not storing it in database may incur much computation cost. Deciding when a property should be stored persistently is a general problem that has not been studied in prior work.


%Not all applications make the best decision on whether to physically store a field.
For example, when we profile the latest version of Openstreetmap~\cite{openstreetmap}, a collaborative editable map system, 
we find that a lot of time is spent on generating a \texttt{location\_name} string for every diary based on the diary's longitude, latitude, and language properties stored in the \texttt{diary\_entry} table. Such slow computation results in a problematic action taking 1 second to show only 20 diaries.
However, the \texttt{location\_name} is usually a short string and remains the same value since the location information for a diary changes infrequently. Storing this string physically as a database column avoids the expensive computation. We evaluate this optimization and find it reducing the action time to only 0.36 second.
%we find its 
%\texttt{diary\_entry/index} action problematic, taking 1.00 seconds to list the text abstract of only 20 diaries. 
%We then find that most of the time is spent in generating a \texttt{location\_name} string for every diary based on the diary's longitude, latitude, and language properties stored in the \texttt{diary\_entry} table. 

%For example, given 
%\{57.7089$^{\circ}$N, 11.9746$^{\circ}$E, English\},
%``Gothenburg, Sweden'' will be computed for display.
%This computation is not cheap, and the size of the \texttt{location\_name} string is negligible for every diary. Furthermore, the location information of a diary almost never changes once created and tends to be displayed for many times.
%Once we add a \texttt{location\_name} column into the \texttt{diary\_entry} table, the server time of \texttt{diary\_entry.index} drops from 1.00 seconds to 0.36 seconds. 

%longi, lati, scale, language
%diary entry.index
%text list
%

We observe similar problems in the bug reports of Lobster, Spree, and Fallingfruit, and in the latest version of Redmine, Fallingfruit, and Openstreetmap. Clearly, developers need help on 
performance estimation to determine which fields to persistently store in database tables. 
%We outline this as part of future research problems in Section \ref{sec:dis}.

%\cong{Interesting. So whether to materialize a field matters to the performance, right? But seeing from the table there is only one case. Will you be able to find more examples?}
\begin{comment}
\subsubsection{To Associate or Not to Associate (Missing Associations)} 
\label{sec:addassoc}

Rails, like other ORM frameworks, allows developer to declare whether two %persistently stored 
classes have no relationships, or have one or multiple
\texttt{OneToOne}, \texttt{OneToMany}, \texttt{ManyToOne} association relationship(s) between them. Determining the association relationships is another hard task for developers. 
%as different designs have different implications for the performance of the resulting queries along with storage costs. 
\shan{TODO: rewrite the next few sentences.}
Prior work~\cite{bag:tradeoff} attempts to reconcile the trade-off between time and space performance among different object-relational mapping strategies. However, this only helps when an application is designed from scratch. For mature applications, 
it will takes a lot of effort to find the missed association.


Figure \ref{fig:spree7511} shows an example of developers adding a \texttt{has\_one} association to improve performance (\texttt{Spree-7511}).
At first, developers discovered an 
N+1 query problem (Section
\ref{sec:iffidata}) in a view file
\texttt{\_order\_details.html.erb}. As shown in Figure \ref{fig:spree7511}, this view file first loads an array of {\tt shipments} from the {\tt shipment} table, and then, for each individual shipment object \texttt{sm} in this array,
a \texttt{selected\_shipping\_rate} function is invoked to issue a 
\texttt{SELECT} query to the {\tt ShippingRate} table.
In order to batch these N+1 queries to 
{\tt shipment} table and {\tt ShippingRate} table together into one query, developers decided to add
\texttt{selected\_shipping\_rate} as an association between these two tables/model-classes, denoted by the added
\texttt{has\_one} line in Figure \ref{fig:spree7511}. 
This new association allows all the shipments' 
selected shipping rates to be loaded in one query using the \texttt{includes} shown in the patch of \texttt{\_order\_Details.html.erb}. 

Note that, adding an association usually requires adding one table's primary key
into the other table as a foreign key, which incurs extra storage cost. 
\cong{I doubt that extra storage cost is the concern. I think maybe the complicated conditional association is hard to use/not intuitive?}
In this example, since there was already a \texttt{has\_many}
association called \texttt{shipping\_rates} between these two classes,
the newly added association incurs almost no cost.

We have observed similar problems and patches in the bug-tracking
systems of Lobster, Spree, and Diaspora. 
We also found a similar problem in the latest version of Tracks. By
adding an extra association, the end-to-end latency of Tracks' \texttt{projects/review} drops
from 0.98 seconds to 0.65 seconds.

%\shan{need a better explanation. maybe start from explicitly explaining $N+1$ problems. maybe we can mention class cohesion ... point out that this is not a db problem}
\begin{figure}[h]
  \centering
  \includegraphics[width=1\columnwidth]{figs/spree7511}
  \caption{Adding association to enable eager loading (Spree)}
  \label{fig:spree7511}
  
\end{figure}
\end{comment}
\vspace{-0.10in}
\subsubsection{Missing Database Indexes (MI)}

Having the appropriate indexes on tables is important for query processing and is a 
well-studied problem \cite{Ullman:1997}. 
As shown in Table \ref{tab:freq}, missing index is the most common performance problem reported in ORM application's bug tracking systems. However, it only appears in three out of the \numpactions problematic actions in latest versions.
We speculate that ORM developers
often do not have the expertise to pick the optimal indexes at the design
phase and hence add table indexes in an incremental way depending on
which query performance becomes a problem after deployment.
%In most of the missing-index issues, an index was added to improve the performance of join queries (7 out of 11), particularly join queries whose \texttt{join} clause contains more than 10 columns (xx out of xx) Missing single and multi-column indexes are both common (about 2:1 ratio)
%.
%missing indexes were so common for several reasons. 
%First, developers of database-backed web applications are often not database optimization experts 
%(partially due to the ORM hiding the database away as an abstraction), and picking the optimal indexes requires understanding the semantics of the application and also how each call to the ORM framework is translated to queries.
%Furthermore, even if developers have the required expertise, they might worry about the cost (in terms of disk and memory space) for creating indexes.
%especially when a table already contains several other indexes. xxx. 
%In fact, xx out of xx missing-index tables in our study were initially released with only one index on its primary key, and only when query performance becomes a problem after deployment will indexes be added in subsequent versions of the application.




\subsection{Application Design Trade-offs}
\label{sec:appdesign}

Developers fix 33 out of the \numissues issue reports by adjusting application display or removing costly functionalities. 
We find similar design problems in latest
versions of 7 out of 12 ORM applications. It is impractical to
completely automate display and functionality design. 
However, our study
shows that ORM developers need tool support, which does not exist yet, to be more informed about the performance implication of their application design decisions.
%found no effectively way to solve the performance problems unless developers throw away some unworthy functionality.
\vspace{-0.10in} 
\subsubsection{Content Display Trade-offs (DT)}
In our study, the most common cause for scalability problems is that a controller action displays \textit{all} database records satisfying certain condition in one page. When the database size increases, the corresponding page takes a lot of time to load due to the increasing amount of data to retrieve and render. This problem contributes to 15 out of the
34 problematic actions that do not scale well in our study. It also appears in 7 out of \numissues issue reports, and is \textit{always} fixed by pagination, i.e., display only a fixed number of records in one page and allow users to navigate to remaining records. 

%As an example, the \texttt{products/index} page in \texttt{Ror\_ecommerce} renders all \textit{products} that are stored in the database. As the number of products increases, the time taken to render the page will increase as well. In practice, users often do not need to see all contents within one page. (e.g., they may only want to see the most popular products). By rendering products by pagination (showing only \alvin{XXX} per page), we find that performance can be improved by $5\times$.

\textcolor{black}{ For example, in \texttt{Diaspora-5335} %named ``\texttt{Paginate cont-}\texttt{acts}'', 
developers used the \texttt{will\_paginate} library~\cite{gem:paginate} to render 25 contacts per page and allow users to see the remaining contacts by clicking the navigation bar at the bottom of the page, instead of showing all contacts within one page as in the old version.
%\texttt{Redmine}, on the other hand, solves a similar problem in 
%\texttt{Redmine-5286} by letting users specify how many issues they want to see within one page.
Clearly, good UI designs can both enhance user experience and improve application performance.} 

%Pagination 
%is widely used in webpage displaying; there are also Rails library support for implementing pagination \cite{gem:paginate}. 
Unfortunately, the lack of pagination still widely exists in latest versions
of ORM applications in our study. This indicates that ORM developers need 
database-aware performance-estimation 
support to remind them of the need to use pagination in webpage design.
\vspace{-0.08in}

\subsubsection{Application Functionality Trade-offs (FT)}
\label{sec:simplifyfeatures}
%When designing the application, without knowing the real workload, it's hard for designers to know the exact time consumed on certain functional feature. As a result, the application may contain some features that cause severe performance problems without bringing much functionality appealing.
%This is particularly a problem in ORM applications, because it is often difficult for developers to know whether and what queries are issued underlying Ruby code, not to mention estimating the query cost.

It is often difficult for ORM developers to estimate performance of a new application feature given that they need to know what queries will be issued by the ORM, how long these queries will execute, and how much data will be returned from the database. In our study, all but two applications have performance issues fixed by developers through removing functionality. 

For example, \texttt{Tracks-870} made a trade-off between performance and functionality by removing a sidebar on the resulting page. This side bar retrieves and displays all the projects and contexts of the current user, and costs a lot of time for users who have participated in many projects.
In the side-bar code, the only data-related part is simply a \texttt{@sidebar.active\_projects} expression, which seems like a trivial
heap access but actually issues a \texttt{SELECT} query and retrieves a lot of data from the database.

As another example, our profiling finds that the \texttt{story.edit} action in the latest version of Lobsters takes 1.5 seconds just to execute one query that determines whether to show the \texttt{guidelines} for users when they edit stories, while the entire page takes 2 seconds to load altogether. Since the  \texttt{guidelines} object only takes very small amount of space to show on the resulting page, removing such checking has negligible impact to the application functionality, yet it would speed up the loading time of that page a lot. 


%\texttt{sidebar} to show the \texttt{actions} and \texttt{contexts} and finally decide the remove \texttt{sidebar}.

In general, performance estimation for applications built using ORMs is important yet has not been done before. It is more difficult as compared to traditional applications due to multiple layers of abstraction. 
We believe combining static analysis with query scalability estimation \cite{armbrust:sigmod13:scale, fan:pods2014:scale} will help developers estimate application performance, as we will discuss in Section \ref{sec:dis}. 
%estimate performance and scalability of ORM code snippets. Including that feature in IDE could greatly help developers in their ORM software design. 


\begin{comment}
\subsection{Potential Solutions}


\subsubsection{Ruby Code Optimization}

\textbf{use more efficient API}
\textbf{use eager loading to prefetch the data}
\textbf{partial to template}
\subsubsection{DB optimization}

\textbf{Add index}
\textbf{Table denormalization}
\subsubsection{Ruby code re-writing}
\textbf{ Paginating}
\end{comment}






\section{Fixing the Inefficiencies}
\label{sec:opt}

After identifying the performance inefficiencies in the \numpactions problematic actions across the 12 studied applications, we manually fix each of them and measure how much our fixes improve the performance of the corresponding application webpages. Our goal is to quantify the importance of the anti-patterns discussed in Section \ref{sec:causes}. 

\subsection{Methodology}
We use the same 20,000-record database configuration used in profiling to measure performance improvement.
For a problematic action that contains multiple inefficiency problems, we fix one at a time and report the speedup for each individual fix.
To fix API-use problems, we change model/view/control files that are related to the problematic API uses; to add missing indexes or fields, we change corresponding Rails migration files; to apply pagination, we use the standard \texttt{will\_paginate} library~\cite{gem:paginate}. We carefully apply  fixes to make sure we do not change the program semantics.
Finally, for two actions in Lobster, we eliminate the expensive checking about whether to show user guidelines, as discussed in Section \ref{sec:simplifyfeatures}.
%We will accumulate the optimization step by step. For example, $HomeController.recent$ action is slowed down by two redundant computation called \texttt{r1} and \texttt{r2}, after optimization \texttt{o1} which removes removing \texttt{r1}, the execution time is reduced from 3782 to 2045, and after optimization \texttt{o2} which removes \texttt{r2}, the execution is reduced from 2045 to 1013. As a result, the speed-up of \texttt{o1} is 1.8 $\times$, and the speed-up of \texttt{o2} is 2 $\times$. We will count \texttt{o1} and \texttt{o2} separately in Figure \ref{speedup}.
\begin{figure}
\centering
\label{fig:sl}
\begin{subfigure}
% \subfigure[Server-time speedup ($\times$)]{
\centering
    \includegraphics[width=0.4\linewidth]{hownotto/speedup}
   \caption{Server-time speedup ($\times$)}
    \label{fig:speedup}
\end{subfigure}
% }
% \subfigure[Line of code changes]{
\begin{subfigure}
    \centering
    \includegraphics[width=0.4\linewidth]{hownotto/loc}
   \caption{Line of code changes}
    \label{fig:loc}       
\end{subfigure}
% }
\caption{Performance fixes and LOC involved}
\end{figure}

\subsection{Results}
In total, \numacissues fixes are applied across 39 problematic actions \footnote{Among the 40 problematic actions identified by our profiling, 1 of them (from GitLab) spends most of its time in file-system operations and  cannot be sped up unless its core functionality is modified. 
} to solve the \numacissues problems listed in Table \ref{tab:freq}.
\vspace{-0.08in} 
\paragraph{\bf Speedup of the fixes}
Figure \ref{fig:speedup} shows the amount of server-time speedup and the sources of the speedup broken down into different anti-patterns as discussed in Section \ref{sec:causes}.

Many fixes are very effective. About a quarter of them achieve more than 5$\times$ speedup, and more than 60\% of them achieve more than 2$\times$ speedup. Every type of fixes
has at least one case where it achieves more than 2$\times$ speedup. 
%Although never providing more than 5X speedups, changing loading strategy provides more than 4X speedups in four cases. 
The largest speed-up is around 39 $\times $ achieved by removing unnecessary feature in \texttt{StoriesController.new} action in Lobsters, i.e., the example we discussed in Section \ref{sec:simplifyfeatures}. 

There are 40 fixes that alter neither the display nor the functionality of the original application. That is, they fix the anti-patterns discussed in Section \ref{causes:api} and \ref{causes:db}. They
achieve
an average speedup of 2.2$\times$, with a maximum of 9.2$\times$ speedup by adding missing fields in \texttt{GanttsController.show} from Redmine.

For all 39 problematic actions, many of which benefit from more than one fix, their average server time is reduced from \serverbefore seconds to \serverafter seconds, and the corresponding end-to-end page loading time is reduced from \eoebefore seconds to \eoeafter seconds, including client rendering and network communication. In other words, by writing code that contains the anti-patterns discussed earlier, developers degrade the performance of their applications by about $6\times$.
%\shan{we probably also should report the total server time and end-to-end time of the 44 actions before and after the optimization. do you have that figure?}

We have reported these 64 fixes to corresponding developers. So far, we have received developers' feedback for \numconfirmedpaissues of them, all of which have been confirmed to be true performance problems and \numfixedpaissues have already been fixed based on our report.%\shan{TODO: discuss why many have no feedback?}\junwen{For those related to API design, maybe it's hard for developers to give answer soon}
\vspace{-0.08in} 
\paragraph{\bf Simplicity of the fixes}
Figure \ref{fig:loc} shows the lines of code changes required to implement the fixes. 
The biggest change takes 56 lines of code to fix (for an inefficient rendering (IR) anti-pattern), while the smallest change requires only 1 line of code in 27 fixes. More than 78\% of fixes require fewer than 5 lines. In addition, among the fixes that improve performance by 3$\times$ or more, more than 90\% of them take fewer than 10 lines of code. Around 60\% of fixes are intra-procedural, involving only one function.  
%\shan{hmm, the lines of code are actually bigger than i expected. need discussion.}\shan{please also report inter-procedural changes vs intra-procedural changes.} 
\begin{comment}
\begin{figure}[h]
  \centering
  \includegraphics[width=1\columnwidth]{figs/speedup}
  \caption{Application server-time speedup after fixes\shan{to fix anti-pattern captions later, same for next figure}\junwen{done}
  }

  \label{speedup}
\end{figure}

\begin{figure}[h]
  \centering
	\includegraphics[width=1\columnwidth]{figs/loc}
  \caption{Lines of code in performance fixes \alvin{move legend into the graph? I think we have room}}

  \label{locchange}
\end{figure}
\end{comment}


These results quantitatively show that there is still a huge amount of inefficiency in 
real-world ORM applications. Much inefficiency can be removed through few lines of code changes.
 A lot of the fixes can potentially be automated, as we will discuss in Section \ref{sec:dis}.

\begin{comment}
\subsection{Automate Solutions}
\subsubsection{How complex are the changes that developers apply to optimize their programs}

\textbf{  Loc change}
\textbf{  program complexity change}

\subsubsection{Are there common patterns can be automated}

\end{comment}

\section{Finding more API misuses}
\label{sec:staticChecker}
\newcommand*\circled[1]{\tikz[baseline=(char.base)]{
            \node[shape=circle,draw,inner sep=1.5pt] (char) {#1};}}
 \newcommand*\smallcircled[1]{\tikz[baseline=(char.base)]{
            \node[shape=circle,draw,inner sep=1pt] (char) {#1};}}
            
Some problems described in Section~\ref{causes:api} are about simple API misuses. We identify 9 such simple misuse patterns, as listed in Table \ref{tab:badapi}, and implement a static analyzer to search for their existence in latest versions of the 12 ORM applications. 
Due to space constraints, we skip the implementation details. 
%through regular expression matching. 
To recap, these 9 API patterns cause performance losses due to ``An Inefficient Query'' 
({\footnotesize \circled{1}, \circled{2}, \circled{3}}), 
``Moving Computation to the DBMS''
({\footnotesize \circled{7}, \circled{8}, \circled{9}}),
``Moving Computation to the Server''
({\footnotesize \circled{5}}),
``Inefficient Updating''
({\footnotesize \circled{4}}), and
``Unnecessary Data Retrieval''
({\footnotesize \circled{6}}), as discussed in Section~\ref{causes:api}. 

\begin{table}
\vspace{-0.05in}
\centering
\caption{API misuses we found in the latest versions}
{\small
\label{tab:badapi}
\begin{tabular}{l@{\hspace{0.01in}}rrrrrrrrrr}
\toprule
App. & \circled{1}  & \circled{2}   & \circled{3} & \circled{4} & \circled{5}  & \circled{6}  & \circled{7}  & \circled{8} & \circled{9} & SUM \\
\midrule
Ds      & 8  & 61  & 0 & 0 & 6  & 6  & 3  & 0 & 1 & 85  \\
\midrule
Lo      & 1  & 38  & 0 & 0 & 0  & 5  & 1  & 0 & 0 & 45  \\
\midrule
Gi      & 7  & 3   & 0 & 1 & 6  & 3  & 3  & 0 & 0 & 23  \\
\midrule
Re      & 3  & 32  & 0 & 1 & 16 & 7  & 0  & 0 & 0 & 59  \\
\midrule
Sp      & 2  & 10  & 0 & 0 & 0  & 0  & 7  & 1 & 0 & 20  \\
\midrule
Ro      & 0  & 7   & 0 & 1 & 1  & 0  & 2  & 0 & 0 & 11  \\
\midrule
Fu      & 0  & 0   & 0 & 0 & 2  & 0  & 0  & 0 & 0 & 2   \\
\midrule
Tr      & 4  & 22  & 0 & 1 & 3  & 0  & 0  & 0 & 0 & 30  \\
\midrule
Da      & 5  & 42  & 1 & 1 & 0  & 8  & 0  & 0 & 0 & 57  \\
\midrule
On      & 10 & 60  & 0 & 0 & 6  & 0  & 0  & 0 & 0 & 76  \\
\midrule
FF      & 2  & 0   & 0 & 2 & 0  & 0  & 0  & 0 & 0 & 4   \\
\midrule
OS      & 0  & 12  & 0 & 0 & 2  & 2  & 0  & 0 & 0 & 16  \\
\midrule
SUM     & 42 & 287 & 1 & 7 & 42 & 31 & 16 & 1 & 1 & 428\\
\bottomrule
\end{tabular}
}

{\footnotesize 
\begin{tabular}{ll}
\smallcircled{1}: \texttt{any?} $\Rightarrow$ \texttt{exists?} & \smallcircled{2}: \texttt{where.first} $\Rightarrow$ \texttt{find\_by} \\
\smallcircled{3}: \texttt{*} $\Rightarrow$ \texttt{*.except(:order)} & \smallcircled{4}: \texttt{each.update} $\Rightarrow$ \texttt{update\_all} \\
\smallcircled{5}: \texttt{.count} $\Rightarrow$ \texttt{.size} & \smallcircled{6}: \texttt{.map} $\Rightarrow$ \texttt{.pluck} \\
\smallcircled{7}: \texttt{pluck.sum} $\Rightarrow$ \texttt{sum} & \smallcircled{8}: \texttt{pluck + pluck} $\Rightarrow$ \texttt{SQL-UNION}\\
\multicolumn{2}{l}{\smallcircled{9}: \texttt{if exists? find else create end} $\Rightarrow$ {find\_or\_create\_by}}\\
\end{tabular}
 \\

\par} 
\end{table}

As shown in Table \ref{tab:badapi}, 
every API misuse pattern still exists in at least one application's latest version. Worse, 4 patterns each occur in over 30 places across more than 5 applications. 
We have checked all these 428 places and confirmed each of them. For further confirmation, we posted them to corresponding application's bug-tracking system, and every category has issues that have already been confirmed by application developers.
%\junwen{For each pattern, we have received confirmation from at least one application developers that what we found are indeed API misuses.} 
\numconfirmedapi API misuses have been confirmed, and \numfixedapi already fixed in their code repositories based on our bug reports. None of our reports has been denied. %Furthermore, \circled{1} and \circled{5} have already been merged to their repository in Onebody.

Only 3 out of these 428 API misuses coincide with the 64 performance problems listed in Table~\ref{tab:freq} and fixed in Section~\ref{sec:opt}. This is because most of these 428 cases do not reside in the 40 problematic actions that we have identified as top issues in our profiling. However, they do cause unnecessary performance loss, which could be severe under workloads that differ from those used in our profiling.

In sum, the above results confirm our previously identified issues, and furthermore indicate that simple API misuses are pervasive across even the latest versions of these ORM applications.
Yet, there are many other types of API misuse problems discussed in Section \ref{causes:api} that cannot be detected simply
through regular expression matching and will require future research to tackle.

\begin{comment}

\begin{table}
\caption{API misuses in latest versions}
\label{tab:badapi}
{
\begin{tabular}{rl}
\toprule
Misuse Pattern & Applications\\
\midrule
\texttt{count(*)>0} $\Rightarrow$ \texttt{exists?} & {Ds$_9$, Lo$_1$, Gi$_7$, Re$_3$, Sp$_2$, Tr$_4$, Da$_5$, On$_10$, FF$_2$}\\
\texttt{where.first} $\Rightarrow$ \texttt{find\_by} & {}\\
\texttt{*} $\Rightarrow$ \texttt{*.except(:order)} & \\
\texttt{each.update} $\Rightarrow$ \texttt{update\_all} & \\
\texttt{.count} $\Rightarrow$ \texttt{.size} & Red$_3$\\
\texttt{.map} $\Rightarrow$ \texttt{.pluck} & Red$_3$\\
\texttt{.map} $\Rightarrow$ \texttt{.pluck} & Red$_3$\\
\texttt{pluck.sum} $\Rightarrow$ \texttt{sum} & Red$_3$\\
\texttt{array + array} $\Rightarrow$ \texttt{UNION} & Red$_3$\\

\texttt{if exists? } $\Rightarrow$ \&{}\\

\texttt{find\_or\_create\_by} & Red$_3$\\

\bottomrule
\end{tabular}
}
{\footnotesize \\ Subscript: \# of misuses in the latest version of an application \par}
\end{table}
\end{comment}


\section{Discussion}
\label{sec:dis}

%Earlier sections have discussed various performance anti-patterns that widely exist in ORM applications, and also discussed what problems need future research to look at. In the section, we summarize some common themes for future research.

In this section, we summarize the lessons learned and highlight the new research opportunities that are opened up by our study.

%\vspace{-0.05in}
\vspace{-0.09in}
\paragraph{\bf{Improving ORM APIs}}Our study shows that many misused APIs have confusing names, as listed in Table \ref{tab:badapi}, but are translated to different queries and have very different performance. 
%As another example, the {\tt first} query API orders tuples by primary key which is not reflected by the function name and is usually unnecessary.
Renaming some of these APIs could help alleviate the problem.
Adding new APIs can also help developers write well-performing code
without hurting code readability. 
For example, if Rails provides native API support for taking union of 
two queries' results like Django \cite{django} does, there will be
fewer cases of inefficient computation, such as those discussed in Section~\ref{sec:inefficomp}.
%inefficient {\tt pluck(:id)+pluck(:id)} problems.
As another example, better rendering API supports could help eliminate inefficient partial render problem discussed in 
Section~\ref{sec:iffirender}.
%; it applies to the \alvin{?}
To our best knowledge, no ORM framework provides this type of rendering support.
%This also applies to utilizing the database query log and query plan to improving database design like adding missed index.
%\cong{I change the example. bulk create/delete is not a problem in Django.}
%This also applies to the inefficient data accessing problem discussed
%in Section~\ref{sec:iffidata}, where the lack of bulk record creation/deletion/update APIs in Rails inevitably encourages 
%performance problems. New APIs would help eliminate
%such inefficiency \shan{how does other ORM do on this?}. \alvin{I have the same question.}

%For instance, Gitlab uses
%``EXPLAIN ANALYZE'' to check query plan and add index accordingly,
%and such feature may be wanted by many developers since missing
%index is a very common problem and does not have completely automatic
%solution\shan{are we suggesting people all use gitlab tool?}. Another example feature is recursively deleting associate records
%without issueing N+1 query, which is not supported by current API\shan{don't understand. what does this mean?}.
%A third example feature is building template for partial rendering\shan{what does this mean?}.
%Adding an API to support template in extension to existing {\tt %render\_partial}
%API can reduce much redundant computation in rendering similar records.
%\cong{I'm not sure which example is better so I just list all of them here.}
\vspace{-0.12in}
\paragraph{\bf{Support for design and development of ORM applications}}
Developers need help to better understand the performance of their code, especially the parts that involve ORM APIs.
%ORM APIs and the potential cost of ORM code, because currently the time-consuming part of the application is all hidden by the ORM framework. 
%They can no longer just focus on explicit loops in performance estimation.
They should focus on not only loops but ORM library calls (e.g., joins) in performance estimation, since these calls often execute database queries and can be expensive in terms of performance.
Building static analysis tools that can estimate performance 
and scalability of ORM code snippets will alleviate some of the API misuses.
More importantly, this can help developers design 
better application functionality and interfaces, as discussed in Section~\ref{sec:appdesign}. 

Developers will also benefit from tools that can aid in database design, such as suggesting 
fields to make persistent, as discussed in Section~\ref{causes:db}.
While prior work focuses on index design~\cite{autoadmin}, little has been done on aiding developers to determine which fields to make persistent.
%In contrast to traditional database design tools that relies of query logs to infer application semantics~\cite{autoadmin}, 
As the ORM application already contains information on how each object field is computed and used,
%database field is updated and used to compute other properties, this 
this provides a great opportunity for program analysis to further help in both aspects.
%help with database schema design, suggesting which field to be stored persistently.
%suggesting adding or deleting fields.
%this... Traditional design tools also has query log as representative workload. 'query semantic' is in query and need not to infer?}

%Our study indicates that it is possible to build such design-time support.
%recommend the result may better be saved to database.

%\shan{moved here from index}
%Interesting, we notice that Gitlab developers already built a testing tool \cite{sherlock} targeting this problem. This tool leverages postgresql's 
%\texttt{EXPLAIN ANALYZE} functionality to analyze all the query plans 
%of queries issued during testing and see if a column is missing index.
%The above solution is useful but requires
%a large amount of profiling. Future research
%can potentially build static checking tools that xxx.
\vspace{-0.12in} 
\paragraph{\bf{Compiler and runtime optimizations}}
While some performance issues are related to developers' design decisions, we believe that others can be detected and fixed automatically. Previous work has already tackled 
some of the issues such as pushing computation down to database through query synthesis \cite{cheung:pldi13}, query batching \cite{cheung:sigmod14:sloth, dbridge:tkde15}, and avoiding
unnecessary data retrieval \cite{chen:se16:redundantData}. There are still many 
automatic optimization opportunities that remain unstudied. This ranges from checking for API misuses, as we discussed in Section~\ref{sec:staticChecker}, to more sophisticated 
database-aware optimization techniques to remove unnecessary computation (Section~\ref{sec:uncomp}) and inefficient queries (Section~\ref{sec:inefficomp}). 

Besides static compiler optimizations, runtime optimizations or trace-based optimization for ORM frameworks are further possibilities for future research, such as automatic pagination
for applications that render many records,
runtime decisions to move computation between the server and the DBMS, 
runtime decisions to switch between lazy and eager loading,
and 
%can be made whether to perform a computation in memory or to issue a database query, depending on whether the data is already loaded;
runtime decisions about whether to %perform
remove certain expensive functionalities as discussed in Section~\ref{sec:simplifyfeatures}.
%\alvin{I don't get the last one} \cong{is 'remove' better?}
Automated tracing and trace-analysis tools can help model workloads
and workload changes, which can then be used to adapt database and application designs automatically.
%missed foreign key index may be added
%automatically when the system detects that such index will be frequently used if added.
Such tools will need to understand the ORM framework and the interaction among the client, server, and DBMS.
\vspace{-0.13in} 
\paragraph{\bf{Generalizing to other ORM frameworks}}
Our findings and lessons apply to other ORM frameworks as well. The database design (Section \ref{causes:db}) and application design trade-offs (Section \ref{sec:appdesign}) naturally apply across ORMs. Most of the API use problems (Section \ref{causes:api}), like unnecessary computation (UC), data accessing (ID, UD), and rendering (IR), are not limited to specific APIs and hence are general.
While the API misuses listed in Table~\ref{tab:badapi}
may appear to be Rails specific, 
there are similar misuses in applications built upon Django ORM~\cite{django} as well: {\tt exists()} is more efficient than 
\texttt{count>0} ({\footnotesize \circled{1}}); {\tt filter().get()} is faster than 
\texttt{filter().first} ({\footnotesize \circled{2}}); {\tt clear}
\texttt{\_ordering(True)} is like 
\texttt{except(:order)} ({\footnotesize \circled{3}}); {\tt all.update} can batch updates ({\footnotesize \circled{4}}); 
{\tt len()} is faster than {\tt count()} with loaded arrays ({\footnotesize \circled{5}});  
 {\tt only()} is like {\tt pluck()}({\footnotesize \circled{6}}); 
  {\tt aggregate}
  {\tt (Sum)} is like {\tt sum} in Rails  ({\footnotesize \circled{7}}); 
   {\tt union} allows two query results to be unioned in database ({\footnotesize \circled{8}}); 
{\tt get\_or\_create} is like 
{\tt find\_or\_create\_by} in Rails 
({\footnotesize \circled{9}}). 
We sampled 15 issue reports each from top 3 popular Django applications on GitHub. As shown below, these 45 performance issues fall into the same 8 anti-patterns our \numissues Rails issue reports fall into: 
%Furthermore, for {\bf every} anti-pattern observed in Rails issue reports, we also observed it in issue reports from top 3 popular Django applications on GitHub \cite{redash,zulip,django-cms}:
%IC \cite{zulip3629}, UC \cite{cms4461}, ID \cite{cms1117}, UD \cite{zulip375}, MF \cite{cms912}, MI \cite{zulip5214}, DT \cite{redash625}, FT \cite{redash300} (many more Django examples skipped for space).

\begin{table}[h]
\centering
\footnotesize
%\caption{Number of problematic actions in each application}
\begin{tabular}{@{\hspace{0.1in}}l@{\hspace{0.1in}}l@{\hspace{0.1in}}l@{\hspace{0.1in}}l@{\hspace{0.1in}}l@{\hspace{0.1in}}l@{\hspace{0.1in}}l@{\hspace{0.1in}}l@{\hspace{0.1in}}l@{\hspace{0.1in}}}
\toprule
       & IC& UC& ID& UD& MF& MI& DT & FT\\
\midrule
Redash \cite{redash} 		& 2 & 3 & 6 & 0 & 0 & 0 & 2 & 2\\
Zulip \cite{zulip}  		& 2 & 5 & 2 & 1 & 0 & 2 & 1 & 2 \\
Django-CMS \cite{django-cms}& 0 & 9 & 3 & 0 & 1 & 0 & 1 & 1 \\
 \bottomrule
\end{tabular}
\end{table}


%TODO Out of Space
%\paragraph{\textbf{Readability vs. Efficiency}} Developers' initial reaction to the patch in Figure \ref{fig:tracks63} is that it hurts code readability. Patches to xxx actually introduce message-chain code smell. some redundant computation may be caused by fear of hurting code modularity. count vs size patch may not be optimal given the code modularity. \shan{why some problems exist in old version and still appear in latest version? }



\vspace{-0.05in}
\section{Performance}
\label{sec:related}

\vspace{-0.025in}
\paragraph{\bf{Empirical studies}}
Previous work confirmed that performance bugs are prevalent in open-source C/Java programs
and often take developers longer time to fix than other types of bugs~\cite{song.pldi12, zaman.msr12}. Prior work~\cite{mark:se17:javascript} studied the performance issues in JavaScript projects. We target performance problems in ORM applications that are mostly related to how application logic interacts with underlying database and are very different from those in general purpose applications. 
Our recent work \cite{yan:cikm17} looked into the database performance of ORM applications and discussed how better database optimization and query translation can improve ORM application performance. No issue report study or thorough profiling was done. In contrast, our paper performs a comprehensive study on all types of performance issues reported by developers and discovered using profiling. Unnecessary data retrieval (UD), content display trade-offs (DT), and part of the inefficient data accessing (ID) anti-patterns \textcolor{black}{ are the only overlap between this study and our previous work \cite{yan:cikm17} }.
%In our previous work \cite{yan:cikm17}, we looked into the database queries of these applications and xxx\shan{Cong, can you extend?} \alvin{is this double blinded? If so we can't say in our previous work}.\junwen{It's double blind review} In this paper, we perform a thorough study on all performance issues, reported by developers and discovered by our profiling.
%In fact, only xx, xx, and xx anti-patterns overlap between our study and the previous work \cite{yan:cikm17}.

\vspace{-0.08in} 
\paragraph{\bf{Inefficiencies in ORM applications}}
Previous work has addressed specific performance problems in ORM applications, such as locating unneeded column data retrieval~\cite{chen:se16:redundantData}, N+1 query
\cite{chen:se14:antipattern}, pushing computation to the DBMS~\cite{cheung:pldi13}, and query batching~\cite{cheung:sigmod14:sloth, dbridge:tkde15, quro}. While effective, these tools do not touch on many anti-patterns discussed in our work, like unnecessary computation (UC), inefficient rendering (IR), database designs (MF, MI), functionality trade-offs (FT), and also do not completely address anti-patterns like inefficient computation (IC) and inefficient data accessing (ID). 
%there are still many problems left addressed, such as ``Inefficient Query'', ``Should Not be a Query'', ``Redundant Computation'', ``Constant Query'', ``Inefficient Eager Loading'', and database and application design tradeoff problems.
%, \cite{quro} reorders query in transactions, .
%synthetic workload generation \cite{pavlo}
%\cite{bailis.sigmod15} investigates the concurrency control in Rails applications.
%Plenty of works \cite{taoxie.ase11} \cite{Torlak2012} \cite{shadi.ase11} have been conducted to generate testing loads for DBMS.

%\shan{we need to explain the difference from the CIKM paper?}

\vspace{-0.08in}
\paragraph{\bf{Performance issues in other types of software}}
Much research was done to detect and fix performance
problems in general purpose software
\cite{song.pldi12,toddler.icse13,caramel.icse15,isil.pldi15,bloat.fse08,harry.pldi09,harry.pldi10}.
%, such as performance-sensitive API misuses
%\cite{song.pldi12}, inefficient and redundant loop computation
%\cite{toddler.icse13, caramel.icse15,isil.pldi15}, object bloat
%\cite{bloat.fse08,harry.pldi09}, low-utility data structures
%\cite{harry.pldi10}, etc.
%Some of these problems share common root causes as some of the ORM performance anti-patterns, such as redundant computation in loop.
Detecting and fixing ORM performance anti-patterns require a completely different set of techniques that understand ORM and underlying database queries.
%Redundant computation problem in regular applications \cite{song.pldi12, loop.icse17}.



\label{sec:con}
It is increasingly challenging to develop web applications that can deliver both good functionality and desired performance. We present \Tool, a tool that
helps web developers explore the performance-functionality trade-off space
in their web application design. The \Tool estimator provides developers 
with data-processing cost information for every HTML tag that renders
dynamically generated data, while the \Tool optimizer identifies and automates
view-changing refactoring that can greatly improve performance.
The \Tool interface integrates estimator and optimizer together to enable effective web application design. 


\newpage
\bibliographystyle{ACM-Reference-Format}

\bibliography{sigproc,confs} 

\end{document}
