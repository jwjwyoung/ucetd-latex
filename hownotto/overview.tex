\section{Overview}
\label{sec:overview}
In this chapter, we present a comprehensive two-pronged empirical study on a set of 12 Rails applications representing 6 common categories that exercise a wide range of functionalities provided by the ORM framework and DBMS. The study aims to answer three research questions:

\begin{itemize}
\item {\bf RQ 1}: How well do real-world database-backed web applications perform as the amount of application data increases?
\item {\bf RQ 2}: What are the common root causes of performance and scalability issues in such applications?
%are they related to data processing?
\item {\bf RQ 3}: What are the potential solutions to such issues and can they be applied automatically?
\end{itemize}

We choose Rails as it is a popular ORM framework \cite{hotframework}.
%which will be discussed in detail in Section \ref{sec:methodology}. 
%
%(i.e., forum, collaboration, e-commerce, task management, social network, and map\footnote{The categories is chosen from \cite{category}, most ORM apps falls in the 6 categories we choose.
%
%\shan{I forgot to ask you --- why did you pick these 6 categories?}
%\cong{One reference for categorization I found online is https://en.wikipedia.org/wiki/Category:Web\_applications. My feeling is that most ORM apps falls in the five categories in the left column. If needed I can count the exact number (maybe for top 200 open source Rails/ORM apps)}
%}). \alvin{I think we should keep this short and justify why these categories were picked in 3.1 instead}
%
%After identifying these applications, we first thoroughly profile each application to not only understand but also alleviate performance and scalability problems of the selected applications.%, which are top-2 most popular applications from GitHub for each of the 6 categories above. \alvin{why mention top-2 .. here? Shouldn't that be part of the justification in 3.1?}
%
We carefully examine \numissues fixed performance issues randomly sampled from the bug-tracking systems of these 12 applications. This helps us understand how well these applications performed on real-world data \textit{in the past}, and what types of performance and scalability problems have been discovered and fixed by end-users and developers in \textit{past versions}.

To complement the above study, we also conduct thorough profiling and code review of the latest versions of these 12 applications. This investigation helps us understand how these applications \textit{currently} perform on our carefully synthesized data (to be explained in Section \ref{sec:methodology}),  what types of performance and scalability problems still exist in the \textit{latest versions}, and how they can be fixed.

In terms of findings,
for {\bf RQ1}, our profiling in Section \ref{sec:profiling} shows that, under workload that is no larger than today's typical 
workload, 11 out of 12 studied applications contain pages in their latest versions that take more than two seconds to load and also pages that scale super-linearly. Compared to client-side computation (e.g., executing JavaScript functions in the browser), server-side computation takes more time in most time-consuming pages and often scales much worse. 
These results motivate research to tackle server-side performance problems in ORM applications.

For {\bf RQ2}, we generalize 9 ORM performance anti-patterns by thoroughly studying about 200 real-world performance issues, with \numissues collected from 12 bug-tracking systems and \numacissues manually identified by us based on profiling the same set of ORM applications (Section \ref{sec:causes}).  We group these 9 patterns into three major categories---ORM API misuses, database design problems, and trade-offs in application design.
All but one of these patterns exist both in previous versions (i.e., fixed and recorded in bug-tracking systems) and the latest versions (i.e., discovered by us through profiling and code review) of these applications.
6 of these anti-patterns appear profusely in more than half of the studied applications. While a few of them have been identified in prior work, the majority of these anti-patterns have \textit{not} been researched in the past.

Finally, for {\bf RQ3}, we manually design and apply fixes to the \numacissues performance issues in the latest versions of these 12 ORM applications (Section \ref{sec:opt}). Our fixes achieve 2$\times$ median speedup (and up to \maxspeedup$\times$) in server-side performance improvement, and reduce the average end-to-end latency of 39 problematic web pages in 12 applications from \eoebefore seconds to \eoeafter seconds, making them interactive.
Most of these optimizations follow generic patterns that we believe can be automated in the future through static analysis and code transformations. As a proof of concept, we implement a simple static checker based on our findings and use it to find hundreds of API misuse performance problems in the latest versions of these applications (Section~\ref{sec:staticChecker}).