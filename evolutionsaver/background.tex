\section{Background and Extended Motivation}
\label{sec:back}

\textbf{Background.} 
%ORM frameworks offer a \textit{migration} mechanism to consistently alter the database schema over time. 
A web application's schema gets initialized and 
updated by \textit{migration APIs} in migration 
files, a mechanism supported by ORM frameworks. 
For example, Listing \ref{migration} illustrates
two migration files, each with one migration API
call: the first creates a table named \texttt{people} with two columns \texttt{id} and \texttt{sequence},
which are automatically mapped to two fields in a corresponding model class
with a singular name
(\texttt{Person} class); the second 
renames a table column, which automatically causes
a field name change in its model class. 

During an installation/upgrade of a 
web application, the ORM framework executes all the
latest migration files not yet executed
on this installation, calling migration APIs in these files one by one and updating the schema along the way.
%; an existing installation will only execute the latest migration entry upon its software upgrade.

%\lstinputlisting[basicstyle=\footnotesize\ttfamily, label={migration}, caption={Inconsistent code from Onebody},language=Ruby]{migration.rb}
