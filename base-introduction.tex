Modern web applications face the challenge of processing a growing amount of data while serving increasingly impatient users. On one hand, popular web applications typically increase their user bases by 5--7\% per week in the first few years~\cite{startup}, with quickly growing data that is produced or consumed by these users and is managed by applications. 
%\alvin{does growing number of users always imply mean growing data size? I don't think so. Take news website for instance. Does more viewers mean more news??} \junwen{except for user, we can also show the data for the posts growth for facebook and other applications: Every 60 seconds on Facebook, 510,000 comments are posted, 293,000 statuses are updated, and 136,000 photos are uploaded.\cite{facebook:2}}
%Twitter\cite{twitter} has shown that their monthly active users worldwide have increased from 30 million to 328 million during 2010 to 2017.
On the other hand, studies have shown that nearly half of the users expect a site to load in less than 2 seconds and will abandon a site if it is not loaded within 3 seconds \cite{akamai2011}, while every extra 0.5 second of latency reduces the overall traffic by 20\% \cite{google.2006}.

% ORM 
%\alvin{I suggest add a paragraph here to first briefly describe how database-backed web apps are constructed. This will also be a good time to introduce ORMs, and then say developing apps that are efficient requires cross stack understanding}

To manage large amounts of data, modern web applications often follow a three-stack architecture: (1) a web interface developed by markup language like HTML (2) application logic developed by traditional object-oriented languages, and (3) a database management system (DBMS) that maintains persistent data.
To help developers %who are mostly not database experts 
construct such database-backed web applications, Object Relational Mapping (ORM) frameworks have become increasingly popular, with implementations in all common general-purpose languages: 
the Ruby on Rails framework (Rails) for Ruby \cite{ror}, Django for Python \cite{django}, and Hibernate for Java \cite{hibernate}. These ORM frameworks allow developers to program such database-backed web applications in a DBMS oblivious way, as the frameworks expose APIs for developers to operate persistent data stored in the DBMS as if they are regular heap objects, with regular-looking method calls transparently translated to SQL queries by frameworks when executed.

Under this architecture, performance and correctness problems are common without cross-stack analysis. For example, application developers
may not know what DB queries are issued at run time, DBMS may miss query-optimization opportunities without knowing high-level application logic, and
web designers often have no idea about the DB cost of generating particular HTML tags, which all lead to performance problems. Furthermore, data constraints
and specifications, which are key to the correctness of data-centered applications, are difficult to maintain across stacks, which can lead to severe correctness problems. It is also difficult for developers to keep their code 
consistent with database schema changes all the time,

To tackle these problems, on the performance side, I have conducted a comprehensive study of performance issues in real-world DB-backed web applications~\cite{yang2018not} and developed techniques to tackle each main category of performance problems revealed by my study~\cite{yang:fse18:powerstation,yang2019panorama}, detecting and fixing \textit{thousands} of performance issues in 
latest versions of popular web applications along the way. On the correctness side, I have conducted an in-depth empirical study about data constraint problems in
real-world DB-backed web applications and developed static checkers that identified \textit{thousands} of data constraint issues~\cite{yang2020managing}, which can lead to software crashes and severe failures. I have also conducted an empirical study on the schema change and developed  \ETool{} which detects 
schema-code inconsistency and suggests refactoring in
web applications. 
