Modern web applications have to manage a lot of user data and often use a three-stack architecture: (1) a web interface developed by markup language like HTML (2) application logic developed by traditional object-oriented languages, and (3) a database management system (DBMS) that maintains persistent data.
 
Under this architecture, there are common understanding gaps between web designer and application developers and database engine. As a result, performance and correctness problems are common.
 
To tackle performance problems, we did one of the first studies to understand why real-world web applications are slow. Guided by that study, we used cross-stack analysis to synthesize efficient web pages, to automatically detect inefficiency in data processing code, and to help database optimization using application knowledge. Our tools have found thousands of performance issues. To tackle correctness problems in web applications, we investigated how data constraints could be inconsistent across web interface, application, and database, and how to solve these problems.