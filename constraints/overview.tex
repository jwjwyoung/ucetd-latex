In this chapter, we aim to answer four key research questions about real-world database-backed
web applications, as listed in Table~\ref{table:highlight} by comprehensively
studying the source code, the commit history, and the issue-tracking system of 12 popular Ruby on Rails applications that represent 6 most common web-application categories. 

\begin{table}[h]
\centering
\caption{Highlight results of our study} 
\footnotesize{\textmd{(*: all the identified issues are in latest versions of these applications)}}
 \setlength{\tabcolsep}{1pt} 
\label{table:highlight}
\resizebox{0.7\columnwidth}{!}{
\begin{tabular}{@{}ll@{}}
\toprule
\multicolumn{2}{c}{\bf RQ1: How are constraints specified in one software version?}\\
 
{How}    &  2.1 per 100 LoC            \\
%\cmidrule(l){2-2} 
Many?              &  1.4 per 1 data field       \\ 
% \cmidrule(l){2-2} 
                              & 77\% of data fields have constraints\\
%                              \midrule
\multirow{2}{*}{Where?}       & 76\% in DB; 23\% in application; 1\% in front-end     \\  
%\cmidrule(l){2-2} 
                              & 24\% of application constraints are missing in DB\\
                             
                              \midrule
\multicolumn{2}{c}{\bf RQ2: How are constraints specified across versions?}\\
 
                                & 49\% of versions contain constraint changes  \\
                                & $>$ 25\% of changes tighten constraints on existing data fields\\
                                
\midrule                                 
\multicolumn{2}{c}{\bf RQ3: What led to real-world constraint problems?}\\ 
 
Where       & 21\% of \numissues studied issues\\
What        & 51\% of \numissues studied issues\\
When        & 10\% of \numissues studied issues\\
How         & 18\% of \numissues studied issues\\
\midrule 
\multicolumn{2}{c}{\bf RQ4: Can we identify constraint problems in latest version?}\\
Where       & 1000+ string fields have length constraints in DB but not in app. \\
            & 200+ fields forbidden to be {\tt null} in app. but {\tt null} by default in DB\\  
            & 88 fields required to be unique in app. but not so in DB\\
            & 57 in(ex)clusion constraints specified in app. but missed in DB\\
            & 133 conflicting length/numericality constraints between app. and DB\\ 
What        & 19 incorrect case-sensitivity constraints identified\\
How         & 2 missing error-message problems identified\\
            & API default error-message enhancement preferred in user study\\
\bottomrule
\end{tabular}
}

\end{table}

{For RQ1}, we wrote scripts to collect and compare constraints expressed in various components of the latest versions of the 12 applications. We found that about three-quarter of all data fields are associated with constraints. In total, there are hundreds to over 
one thousand constraints explicitly specified in 
each application, averaging 1.1--3.6 constraints
specified per 100 lines of code. Data presence and data length are the two most common types of constraints,
while complicated constraints like the relationship among multiple fields also exist. 
We also found
that hundreds to thousands of constraints specified in the
database are missing in the application source code, and vice versa, 
which can lead to maintenance, functionality,
and performance problems.
The details are presented in Section \ref{sec:wherewhat}.


{For RQ2}, we checked how data constraints change throughout the applications' development history. 
We found that about 32\% of all the code changes related to data constraints is about adding new constraints or changing existing ones on data fields that have already existed in software. These changes, regardless of whether they are due to developers' earlier mistakes or warranted by new code features, can easily lead to upgrade and usage problems for data that already exists in the database.
The details are in Section \ref{sec:evolve}.
 
{For RQ3}, we thoroughly investigated \numissues real-world issues that are related to data constraints. We categorize them into four major anti-patterns:
(1) inconsistency of constraints specified at different places, which we refer to as the {\it Where} anti-pattern;
(2) inconsistency between constraint specification and actual data usage in the application, which we refer to as the {\it What} anti-pattern;
(3) inconsistency between data/constraints between different application versions,which we refer to as the {\it When} anti-pattern;
and (4) problems with how constraint-checking results are
delivered (i.e., unclear or missing error messages), which
we refer to as the {\it How} anti-pattern. These four anti-patterns are all common and difficult to avoid by developers; they led to a variety of failures such as web-page crashes, silent failures, software-upgrade failures, poor user experience, etc.   
The details are presented in Section \ref{sec:causes}.

{For RQ4}, we developed tools that automatically identify many
data-constraint problems in the latest versions of these 12 applications, as highlighted in Table \ref{table:highlight}.  
We found around 2,000
``Where'' problems, including many fields
that have important constraints specified in the database
but not in the application or vice versa, as well as
over 100 fields that have length or numericality (i.e., numerical type and value range)
constraints specified in both the database and the application, 
but the constraints conflict with each other. We also found 19
issues in which the field is associated with case-insensitive uniqueness constraints, 
but are used by the application in a case-sensitive way
(the ``What'' anti-pattern), as well as
two problems related to missing error messages (the ``How'' anti-pattern).
We manually checked around 200 randomly sampled problems and found a low
false positive rate (0--10\%) across different types of checks. 
Not to overwhelm application developers, 
we reported \numreportedissues of these problems to them, covering all problem categories. We received \numconfirmedissues confirmation from the developers (no feedback yet to the other \numunconfirmedissues reports), among which our proposed patches for \nummergedissues of those problems have already been merged into their applications or included in the next major release. 
 %\shan{fill in xx}

We also developed a Ruby library that improves the default error messages of
five Rails constraint-checking APIs. We performed a user study with results showing that web users overwhelmingly prefer our enhancement.
The details are presented in Section \ref{sec:solution}.
 
Overall, this paper presents the first in-depth study of data constraint problems in web applications.
Our study provides motivations and guidelines for future research to help developers
better manage data constraints.   We have prepared
a detailed replication package for the data-constraint-issue study and the data-constraint checking tools  in this paper. This package is available on the webpage of our open-source Hyperloop project~\cite{hyperloop}, a
project that aims to solve database-related problems
in ORM applications.
